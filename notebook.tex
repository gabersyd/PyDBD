
% Default to the notebook output style

    


% Inherit from the specified cell style.




    
\documentclass[11pt]{article}

    
    
    \usepackage[T1]{fontenc}
    % Nicer default font (+ math font) than Computer Modern for most use cases
    \usepackage{mathpazo}

    % Basic figure setup, for now with no caption control since it's done
    % automatically by Pandoc (which extracts ![](path) syntax from Markdown).
    \usepackage{graphicx}
    % We will generate all images so they have a width \maxwidth. This means
    % that they will get their normal width if they fit onto the page, but
    % are scaled down if they would overflow the margins.
    \makeatletter
    \def\maxwidth{\ifdim\Gin@nat@width>\linewidth\linewidth
    \else\Gin@nat@width\fi}
    \makeatother
    \let\Oldincludegraphics\includegraphics
    % Set max figure width to be 80% of text width, for now hardcoded.
    \renewcommand{\includegraphics}[1]{\Oldincludegraphics[width=.8\maxwidth]{#1}}
    % Ensure that by default, figures have no caption (until we provide a
    % proper Figure object with a Caption API and a way to capture that
    % in the conversion process - todo).
    \usepackage{caption}
    \DeclareCaptionLabelFormat{nolabel}{}
    \captionsetup{labelformat=nolabel}

    \usepackage{adjustbox} % Used to constrain images to a maximum size 
    \usepackage{xcolor} % Allow colors to be defined
    \usepackage{enumerate} % Needed for markdown enumerations to work
    \usepackage{geometry} % Used to adjust the document margins
    \usepackage{amsmath} % Equations
    \usepackage{amssymb} % Equations
    \usepackage{textcomp} % defines textquotesingle
    % Hack from http://tex.stackexchange.com/a/47451/13684:
    \AtBeginDocument{%
        \def\PYZsq{\textquotesingle}% Upright quotes in Pygmentized code
    }
    \usepackage{upquote} % Upright quotes for verbatim code
    \usepackage{eurosym} % defines \euro
    \usepackage[mathletters]{ucs} % Extended unicode (utf-8) support
    \usepackage[utf8x]{inputenc} % Allow utf-8 characters in the tex document
    \usepackage{fancyvrb} % verbatim replacement that allows latex
    \usepackage{grffile} % extends the file name processing of package graphics 
                         % to support a larger range 
    % The hyperref package gives us a pdf with properly built
    % internal navigation ('pdf bookmarks' for the table of contents,
    % internal cross-reference links, web links for URLs, etc.)
    \usepackage{hyperref}
    \usepackage{longtable} % longtable support required by pandoc >1.10
    \usepackage{booktabs}  % table support for pandoc > 1.12.2
    \usepackage[inline]{enumitem} % IRkernel/repr support (it uses the enumerate* environment)
    \usepackage[normalem]{ulem} % ulem is needed to support strikethroughs (\sout)
                                % normalem makes italics be italics, not underlines
    

    
    
    % Colors for the hyperref package
    \definecolor{urlcolor}{rgb}{0,.145,.698}
    \definecolor{linkcolor}{rgb}{.71,0.21,0.01}
    \definecolor{citecolor}{rgb}{.12,.54,.11}

    % ANSI colors
    \definecolor{ansi-black}{HTML}{3E424D}
    \definecolor{ansi-black-intense}{HTML}{282C36}
    \definecolor{ansi-red}{HTML}{E75C58}
    \definecolor{ansi-red-intense}{HTML}{B22B31}
    \definecolor{ansi-green}{HTML}{00A250}
    \definecolor{ansi-green-intense}{HTML}{007427}
    \definecolor{ansi-yellow}{HTML}{DDB62B}
    \definecolor{ansi-yellow-intense}{HTML}{B27D12}
    \definecolor{ansi-blue}{HTML}{208FFB}
    \definecolor{ansi-blue-intense}{HTML}{0065CA}
    \definecolor{ansi-magenta}{HTML}{D160C4}
    \definecolor{ansi-magenta-intense}{HTML}{A03196}
    \definecolor{ansi-cyan}{HTML}{60C6C8}
    \definecolor{ansi-cyan-intense}{HTML}{258F8F}
    \definecolor{ansi-white}{HTML}{C5C1B4}
    \definecolor{ansi-white-intense}{HTML}{A1A6B2}

    % commands and environments needed by pandoc snippets
    % extracted from the output of `pandoc -s`
    \providecommand{\tightlist}{%
      \setlength{\itemsep}{0pt}\setlength{\parskip}{0pt}}
    \DefineVerbatimEnvironment{Highlighting}{Verbatim}{commandchars=\\\{\}}
    % Add ',fontsize=\small' for more characters per line
    \newenvironment{Shaded}{}{}
    \newcommand{\KeywordTok}[1]{\textcolor[rgb]{0.00,0.44,0.13}{\textbf{{#1}}}}
    \newcommand{\DataTypeTok}[1]{\textcolor[rgb]{0.56,0.13,0.00}{{#1}}}
    \newcommand{\DecValTok}[1]{\textcolor[rgb]{0.25,0.63,0.44}{{#1}}}
    \newcommand{\BaseNTok}[1]{\textcolor[rgb]{0.25,0.63,0.44}{{#1}}}
    \newcommand{\FloatTok}[1]{\textcolor[rgb]{0.25,0.63,0.44}{{#1}}}
    \newcommand{\CharTok}[1]{\textcolor[rgb]{0.25,0.44,0.63}{{#1}}}
    \newcommand{\StringTok}[1]{\textcolor[rgb]{0.25,0.44,0.63}{{#1}}}
    \newcommand{\CommentTok}[1]{\textcolor[rgb]{0.38,0.63,0.69}{\textit{{#1}}}}
    \newcommand{\OtherTok}[1]{\textcolor[rgb]{0.00,0.44,0.13}{{#1}}}
    \newcommand{\AlertTok}[1]{\textcolor[rgb]{1.00,0.00,0.00}{\textbf{{#1}}}}
    \newcommand{\FunctionTok}[1]{\textcolor[rgb]{0.02,0.16,0.49}{{#1}}}
    \newcommand{\RegionMarkerTok}[1]{{#1}}
    \newcommand{\ErrorTok}[1]{\textcolor[rgb]{1.00,0.00,0.00}{\textbf{{#1}}}}
    \newcommand{\NormalTok}[1]{{#1}}
    
    % Additional commands for more recent versions of Pandoc
    \newcommand{\ConstantTok}[1]{\textcolor[rgb]{0.53,0.00,0.00}{{#1}}}
    \newcommand{\SpecialCharTok}[1]{\textcolor[rgb]{0.25,0.44,0.63}{{#1}}}
    \newcommand{\VerbatimStringTok}[1]{\textcolor[rgb]{0.25,0.44,0.63}{{#1}}}
    \newcommand{\SpecialStringTok}[1]{\textcolor[rgb]{0.73,0.40,0.53}{{#1}}}
    \newcommand{\ImportTok}[1]{{#1}}
    \newcommand{\DocumentationTok}[1]{\textcolor[rgb]{0.73,0.13,0.13}{\textit{{#1}}}}
    \newcommand{\AnnotationTok}[1]{\textcolor[rgb]{0.38,0.63,0.69}{\textbf{\textit{{#1}}}}}
    \newcommand{\CommentVarTok}[1]{\textcolor[rgb]{0.38,0.63,0.69}{\textbf{\textit{{#1}}}}}
    \newcommand{\VariableTok}[1]{\textcolor[rgb]{0.10,0.09,0.49}{{#1}}}
    \newcommand{\ControlFlowTok}[1]{\textcolor[rgb]{0.00,0.44,0.13}{\textbf{{#1}}}}
    \newcommand{\OperatorTok}[1]{\textcolor[rgb]{0.40,0.40,0.40}{{#1}}}
    \newcommand{\BuiltInTok}[1]{{#1}}
    \newcommand{\ExtensionTok}[1]{{#1}}
    \newcommand{\PreprocessorTok}[1]{\textcolor[rgb]{0.74,0.48,0.00}{{#1}}}
    \newcommand{\AttributeTok}[1]{\textcolor[rgb]{0.49,0.56,0.16}{{#1}}}
    \newcommand{\InformationTok}[1]{\textcolor[rgb]{0.38,0.63,0.69}{\textbf{\textit{{#1}}}}}
    \newcommand{\WarningTok}[1]{\textcolor[rgb]{0.38,0.63,0.69}{\textbf{\textit{{#1}}}}}
    
    
    % Define a nice break command that doesn't care if a line doesn't already
    % exist.
    \def\br{\hspace*{\fill} \\* }
    % Math Jax compatability definitions
    \def\gt{>}
    \def\lt{<}
    % Document parameters
    \title{Plasma Notebook}
    
    
    

    % Pygments definitions
    
\makeatletter
\def\PY@reset{\let\PY@it=\relax \let\PY@bf=\relax%
    \let\PY@ul=\relax \let\PY@tc=\relax%
    \let\PY@bc=\relax \let\PY@ff=\relax}
\def\PY@tok#1{\csname PY@tok@#1\endcsname}
\def\PY@toks#1+{\ifx\relax#1\empty\else%
    \PY@tok{#1}\expandafter\PY@toks\fi}
\def\PY@do#1{\PY@bc{\PY@tc{\PY@ul{%
    \PY@it{\PY@bf{\PY@ff{#1}}}}}}}
\def\PY#1#2{\PY@reset\PY@toks#1+\relax+\PY@do{#2}}

\expandafter\def\csname PY@tok@w\endcsname{\def\PY@tc##1{\textcolor[rgb]{0.73,0.73,0.73}{##1}}}
\expandafter\def\csname PY@tok@c\endcsname{\let\PY@it=\textit\def\PY@tc##1{\textcolor[rgb]{0.25,0.50,0.50}{##1}}}
\expandafter\def\csname PY@tok@cp\endcsname{\def\PY@tc##1{\textcolor[rgb]{0.74,0.48,0.00}{##1}}}
\expandafter\def\csname PY@tok@k\endcsname{\let\PY@bf=\textbf\def\PY@tc##1{\textcolor[rgb]{0.00,0.50,0.00}{##1}}}
\expandafter\def\csname PY@tok@kp\endcsname{\def\PY@tc##1{\textcolor[rgb]{0.00,0.50,0.00}{##1}}}
\expandafter\def\csname PY@tok@kt\endcsname{\def\PY@tc##1{\textcolor[rgb]{0.69,0.00,0.25}{##1}}}
\expandafter\def\csname PY@tok@o\endcsname{\def\PY@tc##1{\textcolor[rgb]{0.40,0.40,0.40}{##1}}}
\expandafter\def\csname PY@tok@ow\endcsname{\let\PY@bf=\textbf\def\PY@tc##1{\textcolor[rgb]{0.67,0.13,1.00}{##1}}}
\expandafter\def\csname PY@tok@nb\endcsname{\def\PY@tc##1{\textcolor[rgb]{0.00,0.50,0.00}{##1}}}
\expandafter\def\csname PY@tok@nf\endcsname{\def\PY@tc##1{\textcolor[rgb]{0.00,0.00,1.00}{##1}}}
\expandafter\def\csname PY@tok@nc\endcsname{\let\PY@bf=\textbf\def\PY@tc##1{\textcolor[rgb]{0.00,0.00,1.00}{##1}}}
\expandafter\def\csname PY@tok@nn\endcsname{\let\PY@bf=\textbf\def\PY@tc##1{\textcolor[rgb]{0.00,0.00,1.00}{##1}}}
\expandafter\def\csname PY@tok@ne\endcsname{\let\PY@bf=\textbf\def\PY@tc##1{\textcolor[rgb]{0.82,0.25,0.23}{##1}}}
\expandafter\def\csname PY@tok@nv\endcsname{\def\PY@tc##1{\textcolor[rgb]{0.10,0.09,0.49}{##1}}}
\expandafter\def\csname PY@tok@no\endcsname{\def\PY@tc##1{\textcolor[rgb]{0.53,0.00,0.00}{##1}}}
\expandafter\def\csname PY@tok@nl\endcsname{\def\PY@tc##1{\textcolor[rgb]{0.63,0.63,0.00}{##1}}}
\expandafter\def\csname PY@tok@ni\endcsname{\let\PY@bf=\textbf\def\PY@tc##1{\textcolor[rgb]{0.60,0.60,0.60}{##1}}}
\expandafter\def\csname PY@tok@na\endcsname{\def\PY@tc##1{\textcolor[rgb]{0.49,0.56,0.16}{##1}}}
\expandafter\def\csname PY@tok@nt\endcsname{\let\PY@bf=\textbf\def\PY@tc##1{\textcolor[rgb]{0.00,0.50,0.00}{##1}}}
\expandafter\def\csname PY@tok@nd\endcsname{\def\PY@tc##1{\textcolor[rgb]{0.67,0.13,1.00}{##1}}}
\expandafter\def\csname PY@tok@s\endcsname{\def\PY@tc##1{\textcolor[rgb]{0.73,0.13,0.13}{##1}}}
\expandafter\def\csname PY@tok@sd\endcsname{\let\PY@it=\textit\def\PY@tc##1{\textcolor[rgb]{0.73,0.13,0.13}{##1}}}
\expandafter\def\csname PY@tok@si\endcsname{\let\PY@bf=\textbf\def\PY@tc##1{\textcolor[rgb]{0.73,0.40,0.53}{##1}}}
\expandafter\def\csname PY@tok@se\endcsname{\let\PY@bf=\textbf\def\PY@tc##1{\textcolor[rgb]{0.73,0.40,0.13}{##1}}}
\expandafter\def\csname PY@tok@sr\endcsname{\def\PY@tc##1{\textcolor[rgb]{0.73,0.40,0.53}{##1}}}
\expandafter\def\csname PY@tok@ss\endcsname{\def\PY@tc##1{\textcolor[rgb]{0.10,0.09,0.49}{##1}}}
\expandafter\def\csname PY@tok@sx\endcsname{\def\PY@tc##1{\textcolor[rgb]{0.00,0.50,0.00}{##1}}}
\expandafter\def\csname PY@tok@m\endcsname{\def\PY@tc##1{\textcolor[rgb]{0.40,0.40,0.40}{##1}}}
\expandafter\def\csname PY@tok@gh\endcsname{\let\PY@bf=\textbf\def\PY@tc##1{\textcolor[rgb]{0.00,0.00,0.50}{##1}}}
\expandafter\def\csname PY@tok@gu\endcsname{\let\PY@bf=\textbf\def\PY@tc##1{\textcolor[rgb]{0.50,0.00,0.50}{##1}}}
\expandafter\def\csname PY@tok@gd\endcsname{\def\PY@tc##1{\textcolor[rgb]{0.63,0.00,0.00}{##1}}}
\expandafter\def\csname PY@tok@gi\endcsname{\def\PY@tc##1{\textcolor[rgb]{0.00,0.63,0.00}{##1}}}
\expandafter\def\csname PY@tok@gr\endcsname{\def\PY@tc##1{\textcolor[rgb]{1.00,0.00,0.00}{##1}}}
\expandafter\def\csname PY@tok@ge\endcsname{\let\PY@it=\textit}
\expandafter\def\csname PY@tok@gs\endcsname{\let\PY@bf=\textbf}
\expandafter\def\csname PY@tok@gp\endcsname{\let\PY@bf=\textbf\def\PY@tc##1{\textcolor[rgb]{0.00,0.00,0.50}{##1}}}
\expandafter\def\csname PY@tok@go\endcsname{\def\PY@tc##1{\textcolor[rgb]{0.53,0.53,0.53}{##1}}}
\expandafter\def\csname PY@tok@gt\endcsname{\def\PY@tc##1{\textcolor[rgb]{0.00,0.27,0.87}{##1}}}
\expandafter\def\csname PY@tok@err\endcsname{\def\PY@bc##1{\setlength{\fboxsep}{0pt}\fcolorbox[rgb]{1.00,0.00,0.00}{1,1,1}{\strut ##1}}}
\expandafter\def\csname PY@tok@kc\endcsname{\let\PY@bf=\textbf\def\PY@tc##1{\textcolor[rgb]{0.00,0.50,0.00}{##1}}}
\expandafter\def\csname PY@tok@kd\endcsname{\let\PY@bf=\textbf\def\PY@tc##1{\textcolor[rgb]{0.00,0.50,0.00}{##1}}}
\expandafter\def\csname PY@tok@kn\endcsname{\let\PY@bf=\textbf\def\PY@tc##1{\textcolor[rgb]{0.00,0.50,0.00}{##1}}}
\expandafter\def\csname PY@tok@kr\endcsname{\let\PY@bf=\textbf\def\PY@tc##1{\textcolor[rgb]{0.00,0.50,0.00}{##1}}}
\expandafter\def\csname PY@tok@bp\endcsname{\def\PY@tc##1{\textcolor[rgb]{0.00,0.50,0.00}{##1}}}
\expandafter\def\csname PY@tok@fm\endcsname{\def\PY@tc##1{\textcolor[rgb]{0.00,0.00,1.00}{##1}}}
\expandafter\def\csname PY@tok@vc\endcsname{\def\PY@tc##1{\textcolor[rgb]{0.10,0.09,0.49}{##1}}}
\expandafter\def\csname PY@tok@vg\endcsname{\def\PY@tc##1{\textcolor[rgb]{0.10,0.09,0.49}{##1}}}
\expandafter\def\csname PY@tok@vi\endcsname{\def\PY@tc##1{\textcolor[rgb]{0.10,0.09,0.49}{##1}}}
\expandafter\def\csname PY@tok@vm\endcsname{\def\PY@tc##1{\textcolor[rgb]{0.10,0.09,0.49}{##1}}}
\expandafter\def\csname PY@tok@sa\endcsname{\def\PY@tc##1{\textcolor[rgb]{0.73,0.13,0.13}{##1}}}
\expandafter\def\csname PY@tok@sb\endcsname{\def\PY@tc##1{\textcolor[rgb]{0.73,0.13,0.13}{##1}}}
\expandafter\def\csname PY@tok@sc\endcsname{\def\PY@tc##1{\textcolor[rgb]{0.73,0.13,0.13}{##1}}}
\expandafter\def\csname PY@tok@dl\endcsname{\def\PY@tc##1{\textcolor[rgb]{0.73,0.13,0.13}{##1}}}
\expandafter\def\csname PY@tok@s2\endcsname{\def\PY@tc##1{\textcolor[rgb]{0.73,0.13,0.13}{##1}}}
\expandafter\def\csname PY@tok@sh\endcsname{\def\PY@tc##1{\textcolor[rgb]{0.73,0.13,0.13}{##1}}}
\expandafter\def\csname PY@tok@s1\endcsname{\def\PY@tc##1{\textcolor[rgb]{0.73,0.13,0.13}{##1}}}
\expandafter\def\csname PY@tok@mb\endcsname{\def\PY@tc##1{\textcolor[rgb]{0.40,0.40,0.40}{##1}}}
\expandafter\def\csname PY@tok@mf\endcsname{\def\PY@tc##1{\textcolor[rgb]{0.40,0.40,0.40}{##1}}}
\expandafter\def\csname PY@tok@mh\endcsname{\def\PY@tc##1{\textcolor[rgb]{0.40,0.40,0.40}{##1}}}
\expandafter\def\csname PY@tok@mi\endcsname{\def\PY@tc##1{\textcolor[rgb]{0.40,0.40,0.40}{##1}}}
\expandafter\def\csname PY@tok@il\endcsname{\def\PY@tc##1{\textcolor[rgb]{0.40,0.40,0.40}{##1}}}
\expandafter\def\csname PY@tok@mo\endcsname{\def\PY@tc##1{\textcolor[rgb]{0.40,0.40,0.40}{##1}}}
\expandafter\def\csname PY@tok@ch\endcsname{\let\PY@it=\textit\def\PY@tc##1{\textcolor[rgb]{0.25,0.50,0.50}{##1}}}
\expandafter\def\csname PY@tok@cm\endcsname{\let\PY@it=\textit\def\PY@tc##1{\textcolor[rgb]{0.25,0.50,0.50}{##1}}}
\expandafter\def\csname PY@tok@cpf\endcsname{\let\PY@it=\textit\def\PY@tc##1{\textcolor[rgb]{0.25,0.50,0.50}{##1}}}
\expandafter\def\csname PY@tok@c1\endcsname{\let\PY@it=\textit\def\PY@tc##1{\textcolor[rgb]{0.25,0.50,0.50}{##1}}}
\expandafter\def\csname PY@tok@cs\endcsname{\let\PY@it=\textit\def\PY@tc##1{\textcolor[rgb]{0.25,0.50,0.50}{##1}}}

\def\PYZbs{\char`\\}
\def\PYZus{\char`\_}
\def\PYZob{\char`\{}
\def\PYZcb{\char`\}}
\def\PYZca{\char`\^}
\def\PYZam{\char`\&}
\def\PYZlt{\char`\<}
\def\PYZgt{\char`\>}
\def\PYZsh{\char`\#}
\def\PYZpc{\char`\%}
\def\PYZdl{\char`\$}
\def\PYZhy{\char`\-}
\def\PYZsq{\char`\'}
\def\PYZdq{\char`\"}
\def\PYZti{\char`\~}
% for compatibility with earlier versions
\def\PYZat{@}
\def\PYZlb{[}
\def\PYZrb{]}
\makeatother


    % Exact colors from NB
    \definecolor{incolor}{rgb}{0.0, 0.0, 0.5}
    \definecolor{outcolor}{rgb}{0.545, 0.0, 0.0}



    
    % Prevent overflowing lines due to hard-to-break entities
    \sloppy 
    % Setup hyperref package
    \hypersetup{
      breaklinks=true,  % so long urls are correctly broken across lines
      colorlinks=true,
      urlcolor=urlcolor,
      linkcolor=linkcolor,
      citecolor=citecolor,
      }
    % Slightly bigger margins than the latex defaults
    
    \geometry{verbose,tmargin=1in,bmargin=1in,lmargin=1in,rmargin=1in}
    
    

    \begin{document}
    
    
    \maketitle
    
    

    
    Numerical Modeling of Dielectric Barrier Discharge(DBD)

    Plasma, also known as the fourth state of matter is the state of matter
which consists of ionized particles(ions/electrons) and neutrals. The
electric and magnetic force are dominant in the plasma and they play an
important role in the its physics. There are different ways of plasma
generation that include heating or applying high electric field. Under
normal condition plasma do not exist freely on nature as the charged
particles try to recombine and form netural particles. Here we discuss
about the numerical simulation of special type of plasma known as
Dielectrci Barrier Discharge (DBD) that has a lot of industrial
importances.

    \subsubsection{1. Introduction to Dielectric Barrier
Discharge}\label{introduction-to-dielectric-barrier-discharge}

Dielectric Barrier Discharge (DBD) also known as silent discharge is a
type of electrical discharge in which at least one of the electrodes is
shielded with a dielectric layer. Dielectric which is insulator in
nature doesn't allow the flow of charged particles between the
electrodes as a result the flux of charged particles towards the
dielectric surface causes them to get diposited there. Dielectric
Barrier Discharge is a good example of non-thermal plasma. In DBD the
ion temperature is very low (close or equal to room temperature) while
the electron temperature is very high. This property is regarded as a
very important property for industrial use because it enables us to use
the plasma for material treatment. The low ion temperature prevents the
material from getting melted or burnt (bulk properties) while the high
electron temperature helps in changing the properties of the surface
being treated.

    \subsubsection{2. Equivalent Experimental
Setup}\label{equivalent-experimental-setup}

\begin{center}\rule{0.5\linewidth}{\linethickness}\end{center}

\begin{verbatim}
  <img style=" float:left; display:inline" src="reactor.jpg" width="300" height="400" alt="bbc news special" />
  <img style=" float:right; display:inline" src="flow.jpg" width="500" height="400" alt="bbc news special" />
\end{verbatim}

The type of setup used for this research is as shown as in the above
figures. The reactor consists of two circular copper electrodes each of
thickness 5 mm and diameter 7 cm. The surface of the electrodes are
finely polished because the presence of even a tiny spike on the surface
would cause arc discharge between the electrodes. Similarly special
attention is paid in making the electrodes parallel to each other
because if the electrodes are not parallel the plasma would be formed
only on the region where the distance between the electrodes is minimum
which would make the measurement full of errors. Both the electrodes are
covered by glass dielectric of thickness 1mm and diameter 8 cm. The size
of dielectric is chosen to be greater than the size of the electrodes in
order to ensure that the metallic surfaces are completely shielded
because the presence of unshielded metallic surface would result to the
generation of secondary electrons due to the collision of ions from
plasma with the metallic surface. As the experiment is being conducted
in argon medium, the entire setup is enclosed by glass cylinder which is
properly insulated against gas-leakage except at the top left corner
where butterfly valve is place which has two openings one of which is
for the gas inlet and the other is for the gas outlet. The outlet of the
butterfly valve can be controlled by rotating its handle. Special
precaution is taken in order to ensure that the gas inside the chamber
contains no other gases other than argon. For that the chamber is
connected to the argon cylinder for several minutes with the outlet
valve completely open so that the air inside the chamber gets displaced.
Before starting the experiment the outlet of the valve is partially
closed in such a manner that the inlet gas neither changes the air
pressure inside the chamber nor makes increases the turbulence inside
it. The cross section of the chamber is made up of transparent
polycarbonate cylinder because the light emitted by the plasma is
measured with absorption spectrometer. The space between two dielectrics
is set to be 3mm. Uniform glow discharge is obtained in that region. The
upper and lower electrodes are connected to a high voltage source . For
this experiment the applied voltage is set to be 1000 volts and the
frequency of the source is 40 kHz alternating current. Lower electrode
is grounded i.e. the value of potential at lower electrode is always set
to zero whereas the value of potential at the upper electrode changes
according to the applied AC voltage. A voltmeter is connected in
parallel to the electrodes in order to measure the value of potential
across it while the discharge is being produced whereas a shunt resistor
is used in series with the reactor. The voltage across the shunt
resistor is also measured and dividing the value of that voltage by the
shunt resistance gives the current in the reactor.

 Photograph of the discharge

    \subsubsection{3. Need of Numerical
Modeling}\label{need-of-numerical-modeling}

It is very useful to make a detailed study of plasma parameters before
using it in the industrial scale so that the outcome of the plasma
treatment on the materials would become predictable. The average value
of plasma parameters like electron/ion densities/ temperatures or
discharge current can be measured using experimental methods. However,
in some instances it is very difficult to measure the value of those
parameters at various region between the electrodes i.e. estimating the
profile of such parameters between the electrodes. For example, in order
to measure the value of electric current at various regions inside a DBD
reactor, the current probe has to be moved very finely between the
electrodes which is very difficult if the size of the reactor is in
micrometer scale because the thickness of the current probe itself is
around 1mm. Numerical simulation turns out to be very useful in such
instances.

    \subsubsection{4. Basic Equations}\label{basic-equations}

The numerical modeling of the physical phenomenon in our DBD reactor has
been done by solving the corresponding mathematical equations using
appropriate numerical methods. The mathematical equations in this model
include the continuity equation that is coupled coupled with Poisson
equa- tion along with the equation of charge accumulation on the surface
of dielectric. As the reactor is symmetric in axial direction the
modeling is done in one dimension only. The differential equations used
on this numerical model are discussed below. \#\#\#\# Continuity
Equation The continuity equation used in this model is based on the
conservation of charge and other neutral species inside the plasma
reactor. Particles inside the plasma reactor are either advected or
diffused or generated/destroyed by corresponding chemical reactions. The
general form of continuity equation used in this model is,

\begin{equation}
\frac{dn}{ dt}+ \nabla  \Gamma = S
\end{equation}

where \(\Gamma\) is the flux of particles given by,

\begin{equation}
\Gamma  =(Z nv-D \nabla n )
\end{equation}

The first term Znv is the advection component that gives the drift of
charged particles in the direction of electric field. The positive ions
in the plasma move towards the electric field while the electrons and
negative ion move away from it. The neutral species(that includes the
excited molecules) show no response to the direction of electric field.
The \(D\nabla n\) is the diffusion term that gives the flow of particles
due the concentration gradient. The particles move from the region of
higher concentration to the region of low concentration until there is
no concentration gradient. Here n is the number density of plasma
species (electron, ion or excited molecules) with charge Z (\(\pm\)for
electrons/ions), v is the velocity of particles which can be calculated
by multiplying the mobility of the charged particle with the electric
field. D is the diffusion coefficient of the particles that gives the
rate at which the particles are diffused in proportion to the
concentration gradient. S is the source/sink term which corresponds to
the rate at which a particle is formed or destroyed according to the
corresponding chemical reaction. The source/sink term is given by the
formula \(S=m_ik_in_in_j\), where m is the mole fraction of the particle
being generated, \(k_i\) is the corresponding reaction rate of the
particle with number density ni when it collides with the target
particle with number density \(n_j\) (neutral molecules). The value of
mobility, diffusion and reaction rate are the function of electric
field. In this simulation the value of these parameters are obtained
from a software named BOLSIG+ {[}ref..{]} that solves Boltzmann equation
for electrons in weakly ionized gas to calculate the electron transport
coefficients and collision rate coefficients which can be used in the
fluid models.

\paragraph{Poisson's Equation}\label{poissons-equation}

The motion of charge inside the plasma reactor causes the variation of
charge at various regions between the electrodes which in return causes
the variation in electric field inside the reactor. The value of
electrostatic potential at various region between the electrodes can be
calculated by solving the Poisson equation. Since the transport
parameters like mobility and diffusion and the source/sink term used in
the continuity equation is the function of electric field, and the
electric field can be calculated as the negative gradient of electric
potential, the Poisson equation has to be solved each time before
solving the continuity equation in the simulation. The Poisson equation
can be expressed as:

\begin{equation}
\nabla ^2 V=\frac{\rho}{\epsilon}
\end{equation}

where V is the electrostatic potential, \(\rho\) is the charge density
which is given by the product of charge of an electron with the net
charge in the grid point inside the plasma reactor and \(\epsilon\) is
the permittivity of medium.

\paragraph{Equation of Charge Accumulation on Surface of
Dielectric}\label{equation-of-charge-accumulation-on-surface-of-dielectric}

As both the electrodes in this research are covered with dielectric,
which is an insulator and cannot conduct the charged particles to the
electrodes, the flux of electric charges towards the dielectric causes
the deposition of charged particles on the dielectric surface. This is
regarded as the important property of DBD because the charge
accumulation helps in limiting the value of current between the
electrodes and makes the discharge more uniform. The equation of charge
accumulation on the surface of dielectric is represented by the
following equations.

\[\frac{d \sigma_e}{dt}=n_e v_e-\sigma_ev{_e^{des}}-\alpha_{rec}\sigma_i\sigma_e\]

and
\[\frac{d \sigma_i}{dt}=(1+\gamma_i)n_iv_i-\alpha_{rec}\sigma_i\sigma_e\]

where \(\sigma_e\) and \(\sigma_i\) are the surface charge densities of
electrons and ions on the dielectric. \(\gamma_i\), \(v_e^{des}\) and
\(\alpha_{rec}\) are secondary electron emission coefficient, electron
desorption frequency and recombination coefficient whose values are
0.01, \(10s^{-1}\) and \(10^{-6} cm^2s^{-1}\) respectively.

    \subsubsection{5. Numerical Simulation}\label{numerical-simulation}

In this modelling all the differential equations are solved using finite
difference method. As a first approach all the differential equations
are written in finite difference form and the program is executed. The
program however becomes unstable and therefore adaptive time stepping is
used using Von Neumann Stability analysis. The program becomes stable
for quite a long time and finally becomes unstable. The adaptive time
stepping makes the simulation very slow as well. Finally implicit solver
is used which gives stable solution for a quite long time stepping as
well making the simulation very fast as compared to other methods.

    \subparagraph{5.1 Computational Domain}\label{computational-domain}

The space between the electrodes consists of three layers; two
dielectrics and one gas layer. Each of these layers is divided into
equal numbers of grid points.

    \begin{Verbatim}[commandchars=\\\{\}]
{\color{incolor}In [{\color{incolor}6}]:} \PY{k+kn}{import} \PY{n+nn}{numpy} \PY{k}{as} \PY{n+nn}{np}
        \PY{k+kn}{import} \PY{n+nn}{sys}
        \PY{k+kn}{import} \PY{n+nn}{matplotlib}\PY{n+nn}{.}\PY{n+nn}{pyplot} \PY{k}{as} \PY{n+nn}{plt}
        \PY{k+kn}{import} \PY{n+nn}{scipy}\PY{n+nn}{.}\PY{n+nn}{sparse}\PY{n+nn}{.}\PY{n+nn}{linalg} \PY{k}{as} \PY{n+nn}{la}
        \PY{k+kn}{import} \PY{n+nn}{scipy}\PY{n+nn}{.}\PY{n+nn}{sparse} \PY{k}{as} \PY{n+nn}{sparse}
        \PY{c+c1}{\PYZsh{}*** Parameters for the of plasma reactor}
        \PY{c+c1}{\PYZsh{}\PYZhy{}\PYZhy{}\PYZhy{}\PYZhy{}\PYZhy{}\PYZhy{}\PYZhy{}\PYZhy{}\PYZhy{}\PYZhy{}\PYZhy{}\PYZhy{}\PYZhy{}\PYZhy{}\PYZhy{}\PYZhy{}\PYZhy{}\PYZhy{}\PYZhy{}\PYZhy{}\PYZhy{}\PYZhy{}\PYZhy{}\PYZhy{}\PYZhy{}\PYZhy{}\PYZhy{}\PYZhy{}\PYZhy{}\PYZhy{}\PYZhy{}\PYZhy{}\PYZhy{}\PYZhy{}\PYZhy{}\PYZhy{}\PYZhy{}\PYZhy{}\PYZhy{}\PYZhy{}\PYZhy{}\PYZhy{}\PYZhy{}\PYZhy{}\PYZhy{}\PYZhy{}\PYZhy{}\PYZhy{}\PYZhy{}\PYZhy{}\PYZhy{}\PYZhy{}\PYZhy{}\PYZhy{}\PYZhy{}\PYZhy{}\PYZhy{}\PYZhy{}\PYZhy{}\PYZhy{}\PYZhy{}\PYZhy{}\PYZhy{}\PYZhy{}\PYZhy{}\PYZhy{}\PYZhy{}\PYZhy{}\PYZhy{}\PYZhy{}\PYZhy{}\PYZhy{}\PYZhy{}\PYZhy{}\PYZhy{}\PYZhy{}\PYZhy{}\PYZhy{}\PYZhy{}\PYZhy{}\PYZhy{}\PYZhy{}\PYZhy{}\PYZhy{}\PYZhy{}\PYZhy{}\PYZhy{}\PYZhy{}\PYZhy{}\PYZhy{}\PYZhy{}\PYZhy{}\PYZhy{}\PYZhy{}\PYZhy{}\PYZhy{}\PYZhy{}\PYZhy{}\PYZhy{}\PYZhy{}\PYZhy{}\PYZhy{}\PYZhy{}\PYZhy{}\PYZhy{}\PYZhy{}\PYZhy{}\PYZhy{}\PYZhy{}}
        \PY{n}{width}\PY{o}{=}\PY{l+m+mf}{5.0}     \PY{c+c1}{\PYZsh{}space between two dielectric in mm}
        \PY{n}{ngrid0}\PY{o}{=}\PY{l+m+mi}{300}     \PY{c+c1}{\PYZsh{}Number of grid points (between two dielectric)}
        \PY{n}{wd1}\PY{o}{=}\PY{l+m+mf}{1.}        \PY{c+c1}{\PYZsh{}width of first dielectric in mm}
        \PY{n}{wd2}\PY{o}{=}\PY{l+m+mf}{1.}        \PY{c+c1}{\PYZsh{}width of second dielectric in mm}
        \PY{n}{dx}\PY{o}{=}\PY{n}{width}\PY{o}{*}\PY{l+m+mi}{10}\PY{o}{*}\PY{o}{*}\PY{p}{(}\PY{o}{\PYZhy{}}\PY{l+m+mi}{3}\PY{p}{)}\PY{o}{/}\PY{p}{(}\PY{n}{ngrid0}\PY{o}{+}\PY{l+m+mf}{1.0}\PY{p}{)}\PY{c+c1}{\PYZsh{}Grid size in meter}
        \PY{n}{nwd1}\PY{o}{=}\PY{n+nb}{int}\PY{p}{(}\PY{n}{wd1}\PY{o}{*}\PY{l+m+mi}{10}\PY{o}{*}\PY{o}{*}\PY{p}{(}\PY{o}{\PYZhy{}}\PY{l+m+mi}{3}\PY{p}{)}\PY{o}{/}\PY{n}{dx}\PY{p}{)}       \PY{c+c1}{\PYZsh{}number of grid points in first dielectric}
        \PY{n}{nwd2}\PY{o}{=}\PY{n+nb}{int}\PY{p}{(}\PY{n}{wd2}\PY{o}{*}\PY{l+m+mi}{10}\PY{o}{*}\PY{o}{*}\PY{p}{(}\PY{o}{\PYZhy{}}\PY{l+m+mi}{3}\PY{p}{)}\PY{o}{/}\PY{n}{dx}\PY{p}{)}       \PY{c+c1}{\PYZsh{}Number of grid points in second dielectric}
        \PY{n}{wd1}\PY{o}{=}\PY{n}{nwd1}\PY{o}{*}\PY{n}{dx}                 \PY{c+c1}{\PYZsh{}Making wd1 as exact multiple of dx}
        \PY{n}{wd2}\PY{o}{=}\PY{n}{nwd2}\PY{o}{*}\PY{n}{dx}                 \PY{c+c1}{\PYZsh{}making wd2 as exact multiple of dx}
        \PY{n}{inelec}\PY{o}{=}\PY{n}{width}\PY{o}{*}\PY{l+m+mi}{10}\PY{o}{*}\PY{o}{*}\PY{p}{(}\PY{o}{\PYZhy{}}\PY{l+m+mi}{3}\PY{p}{)}\PY{o}{+}\PY{n}{wd1}\PY{o}{+}\PY{n}{wd2}\PY{c+c1}{\PYZsh{}total interelectrode separation}
        \PY{n}{ngrid}\PY{o}{=}\PY{n+nb}{int}\PY{p}{(}\PY{n}{ngrid0}\PY{o}{+}\PY{l+m+mi}{2}\PY{o}{+}\PY{n}{nwd1}\PY{o}{+}\PY{n}{nwd2}\PY{p}{)}    \PY{c+c1}{\PYZsh{}total number of grid points(2 dielectrics +gas medium + all edge points)}
        \PY{c+c1}{\PYZsh{}\PYZhy{}\PYZhy{}\PYZhy{}\PYZhy{}\PYZhy{}\PYZhy{}\PYZhy{}\PYZhy{}\PYZhy{}\PYZhy{}\PYZhy{}\PYZhy{}\PYZhy{}\PYZhy{}\PYZhy{}\PYZhy{}\PYZhy{}\PYZhy{}\PYZhy{}\PYZhy{}\PYZhy{}\PYZhy{}\PYZhy{}\PYZhy{}\PYZhy{}\PYZhy{}\PYZhy{}\PYZhy{}\PYZhy{}\PYZhy{}\PYZhy{}\PYZhy{}\PYZhy{}\PYZhy{}\PYZhy{}\PYZhy{}\PYZhy{}\PYZhy{}\PYZhy{}\PYZhy{}\PYZhy{}\PYZhy{}\PYZhy{}\PYZhy{}\PYZhy{}\PYZhy{}\PYZhy{}\PYZhy{}\PYZhy{}\PYZhy{}\PYZhy{}\PYZhy{}\PYZhy{}\PYZhy{}\PYZhy{}\PYZhy{}\PYZhy{}\PYZhy{}\PYZhy{}\PYZhy{}\PYZhy{}\PYZhy{}\PYZhy{}\PYZhy{}\PYZhy{}\PYZhy{}\PYZhy{}\PYZhy{}\PYZhy{}\PYZhy{}\PYZhy{}\PYZhy{}\PYZhy{}\PYZhy{}\PYZhy{}\PYZhy{}\PYZhy{}\PYZhy{}\PYZhy{}\PYZhy{}\PYZhy{}\PYZhy{}\PYZhy{}\PYZhy{}\PYZhy{}\PYZhy{}\PYZhy{}\PYZhy{}\PYZhy{}\PYZhy{}\PYZhy{}\PYZhy{}\PYZhy{}\PYZhy{}\PYZhy{}\PYZhy{}\PYZhy{}\PYZhy{}\PYZhy{}\PYZhy{}\PYZhy{}\PYZhy{}\PYZhy{}\PYZhy{}\PYZhy{}\PYZhy{}\PYZhy{}\PYZhy{}\PYZhy{}\PYZhy{}}
        \PY{n}{volt}\PY{o}{=}\PY{l+m+mf}{1200.0}    \PY{c+c1}{\PYZsh{}Interelectrode voltage (peak not RMS)}
        \PY{n}{gasdens}\PY{o}{=}\PY{l+m+mf}{2.504e25}          \PY{c+c1}{\PYZsh{}number density of gas at NTP (unit: m\PYZca{}\PYZhy{}3)}
        \PY{n}{dt}\PY{o}{=}\PY{l+m+mf}{1.0e\PYZhy{}9} \PY{c+c1}{\PYZsh{}small tyme interval}
        \PY{n}{frequencySource} \PY{o}{=} \PY{l+m+mi}{40000} \PY{c+c1}{\PYZsh{}30KHz}
        \PY{n}{ee}\PY{o}{=}\PY{l+m+mf}{1.6}\PY{o}{*}\PY{l+m+mi}{10}\PY{o}{*}\PY{o}{*}\PY{p}{(}\PY{o}{\PYZhy{}}\PY{l+m+mi}{19}\PY{p}{)} \PY{c+c1}{\PYZsh{}electronic charge}
        \PY{n}{e0}\PY{o}{=}\PY{l+m+mf}{8.54187817}\PY{o}{*}\PY{l+m+mi}{10}\PY{o}{*}\PY{o}{*}\PY{p}{(}\PY{o}{\PYZhy{}}\PY{l+m+mi}{12}\PY{p}{)} \PY{c+c1}{\PYZsh{}epsilon}
        \PY{n}{townsendunit}\PY{o}{=}\PY{l+m+mf}{1.0}\PY{o}{/}\PY{p}{(}\PY{p}{(}\PY{n}{gasdens}\PY{p}{)}\PY{o}{*}\PY{p}{(}\PY{l+m+mi}{10}\PY{p}{)}\PY{o}{*}\PY{o}{*}\PY{p}{(}\PY{o}{\PYZhy{}}\PY{l+m+mi}{21}\PY{p}{)}\PY{p}{)}
        \PY{n}{Kboltz}\PY{o}{=}\PY{l+m+mf}{1.380}\PY{o}{*}\PY{l+m+mf}{10e\PYZhy{}23}
        
        \PY{c+c1}{\PYZsh{}*** Initialization}
        \PY{c+c1}{\PYZsh{}\PYZhy{}\PYZhy{}\PYZhy{}\PYZhy{}\PYZhy{}\PYZhy{}\PYZhy{}\PYZhy{}\PYZhy{}\PYZhy{}\PYZhy{}\PYZhy{}\PYZhy{}\PYZhy{}\PYZhy{}\PYZhy{}\PYZhy{}\PYZhy{}\PYZhy{}\PYZhy{}\PYZhy{}\PYZhy{}\PYZhy{}\PYZhy{}\PYZhy{}\PYZhy{}\PYZhy{}\PYZhy{}\PYZhy{}\PYZhy{}\PYZhy{}\PYZhy{}\PYZhy{}\PYZhy{}\PYZhy{}\PYZhy{}\PYZhy{}\PYZhy{}\PYZhy{}\PYZhy{}\PYZhy{}\PYZhy{}\PYZhy{}\PYZhy{}\PYZhy{}\PYZhy{}\PYZhy{}\PYZhy{}\PYZhy{}\PYZhy{}\PYZhy{}\PYZhy{}\PYZhy{}\PYZhy{}\PYZhy{}\PYZhy{}\PYZhy{}\PYZhy{}\PYZhy{}\PYZhy{}\PYZhy{}\PYZhy{}\PYZhy{}\PYZhy{}\PYZhy{}\PYZhy{}\PYZhy{}\PYZhy{}\PYZhy{}\PYZhy{}\PYZhy{}\PYZhy{}\PYZhy{}\PYZhy{}\PYZhy{}\PYZhy{}\PYZhy{}\PYZhy{}\PYZhy{}\PYZhy{}\PYZhy{}\PYZhy{}\PYZhy{}\PYZhy{}\PYZhy{}\PYZhy{}\PYZhy{}\PYZhy{}\PYZhy{}\PYZhy{}\PYZhy{}\PYZhy{}\PYZhy{}\PYZhy{}\PYZhy{}\PYZhy{}\PYZhy{}\PYZhy{}\PYZhy{}\PYZhy{}\PYZhy{}}
        \PY{n}{ns}\PY{o}{=}\PY{l+m+mi}{2}          \PY{c+c1}{\PYZsh{}Number of species (Taking helium gas, one is electron and another is helium ion)}
        \PY{n}{ndensity}\PY{o}{=}\PY{n}{np}\PY{o}{.}\PY{n}{zeros}\PY{p}{(}\PY{p}{(}\PY{n}{ns}\PY{p}{,}\PY{n}{ngrid0}\PY{o}{+}\PY{l+m+mi}{2}\PY{p}{)}\PY{p}{,}\PY{n+nb}{float}\PY{p}{)} \PY{c+c1}{\PYZsh{}Density of each species in all grid points between dielectric}
        \PY{n}{ncharge}\PY{o}{=}\PY{n}{np}\PY{o}{.}\PY{n}{array}\PY{p}{(}\PY{p}{[}\PY{o}{\PYZhy{}}\PY{l+m+mi}{1}\PY{p}{,}\PY{l+m+mi}{1}\PY{p}{]}\PY{p}{)}  \PY{c+c1}{\PYZsh{}Charge of the each species}
        \PY{n}{netcharge}\PY{o}{=}\PY{n}{np}\PY{o}{.}\PY{n}{zeros}\PY{p}{(}\PY{n}{ngrid}\PY{p}{,}\PY{n+nb}{float}\PY{p}{)} \PY{c+c1}{\PYZsh{}net charge at all grid points used to solve poission equation}
        \PY{n}{potentl}\PY{o}{=}\PY{n}{np}\PY{o}{.}\PY{n}{zeros}\PY{p}{(}\PY{n}{ngrid}\PY{p}{,}\PY{n+nb}{float}\PY{p}{)} \PY{c+c1}{\PYZsh{}potential at each grid points}
        \PY{n}{uu}\PY{o}{=}\PY{n}{np}\PY{o}{.}\PY{n}{zeros}\PY{p}{(}\PY{n}{ngrid}\PY{p}{,}\PY{n+nb}{float}\PY{p}{)} \PY{c+c1}{\PYZsh{}potential at each grid points}
        \PY{n}{efield}\PY{o}{=}\PY{n}{np}\PY{o}{.}\PY{n}{zeros}\PY{p}{(}\PY{n}{ngrid0}\PY{o}{+}\PY{l+m+mi}{2}\PY{p}{,}\PY{n+nb}{float}\PY{p}{)} \PY{c+c1}{\PYZsh{}electric field at each grid points}
        \PY{n}{mobegrid}\PY{o}{=}\PY{n}{np}\PY{o}{.}\PY{n}{zeros}\PY{p}{(}\PY{n}{ngrid0}\PY{o}{+}\PY{l+m+mi}{2}\PY{p}{,}\PY{n+nb}{float}\PY{p}{)}\PY{c+c1}{\PYZsh{}mobility at grid points, 1 row=1 tyepe of gaseous species}
        \PY{n}{difegrid}\PY{o}{=}\PY{n}{np}\PY{o}{.}\PY{n}{zeros}\PY{p}{(}\PY{n}{ngrid0}\PY{o}{+}\PY{l+m+mi}{2}\PY{p}{,}\PY{n+nb}{float}\PY{p}{)}\PY{c+c1}{\PYZsh{}diffusion coefficient at each grid points}
        \PY{n}{sourceegrid}\PY{o}{=}\PY{n}{np}\PY{o}{.}\PY{n}{zeros}\PY{p}{(}\PY{n}{ngrid0}\PY{o}{+}\PY{l+m+mi}{2}\PY{p}{,}\PY{n+nb}{float}\PY{p}{)}\PY{c+c1}{\PYZsh{}reaction rate at each grid points}
        \PY{n}{mobigrid}\PY{o}{=}\PY{n}{np}\PY{o}{.}\PY{n}{zeros}\PY{p}{(}\PY{n}{ngrid0}\PY{o}{+}\PY{l+m+mi}{2}\PY{p}{,}\PY{n+nb}{float}\PY{p}{)}
        \PY{n}{difigrid}\PY{o}{=}\PY{n}{np}\PY{o}{.}\PY{n}{zeros}\PY{p}{(}\PY{n}{ngrid0}\PY{o}{+}\PY{l+m+mi}{2}\PY{p}{,}\PY{n+nb}{float}\PY{p}{)}
        \PY{n}{sig\PYZus{}e\PYZus{}left}\PY{o}{=}\PY{l+m+mf}{0.0}   \PY{c+c1}{\PYZsh{}Electron surface charge density at left dielectric}
        \PY{n}{sig\PYZus{}e\PYZus{}right}\PY{o}{=}\PY{l+m+mf}{0.0}  \PY{c+c1}{\PYZsh{}Electron surface charge density at right dielectric}
        \PY{n}{sig\PYZus{}i\PYZus{}left}\PY{o}{=}\PY{l+m+mf}{0.0}   \PY{c+c1}{\PYZsh{}Ion surface charge density at left dielectric     }
        \PY{n}{sig\PYZus{}i\PYZus{}right}\PY{o}{=}\PY{l+m+mf}{0.0}  \PY{c+c1}{\PYZsh{}Ion surface charge density at right dielectric}
        
        \PY{n}{numberOfSteps} \PY{o}{=} \PY{l+m+mi}{10000000} \PY{c+c1}{\PYZsh{}4100 cycles of the ac freque3ncy}
        
        \PY{c+c1}{\PYZsh{}ndensity = np.ones((2,ngrid0+2),float)*1000.0}
        \PY{n}{ndensity}\PY{o}{=}\PY{l+m+mi}{1000}\PY{o}{*}\PY{n}{np}\PY{o}{.}\PY{n}{random}\PY{o}{.}\PY{n}{rand}\PY{p}{(}\PY{l+m+mi}{2}\PY{p}{,}\PY{n}{ngrid0}\PY{o}{+}\PY{l+m+mi}{2}\PY{p}{)}
\end{Verbatim}


    \subparagraph{5.2 Flow of the program}\label{flow-of-the-program}

    \subparagraph{5.3 Importing the Input transport and reaction
rates}\label{importing-the-input-transport-and-reaction-rates}

    \begin{Verbatim}[commandchars=\\\{\}]
{\color{incolor}In [{\color{incolor}7}]:} \PY{k}{def} \PY{n+nf}{readBoltzmannParameters}\PY{p}{(}\PY{n}{npoints}\PY{p}{,}\PY{n}{oupfile}\PY{o}{=}\PY{l+s+s1}{\PYZsq{}}\PY{l+s+s1}{output.txt}\PY{l+s+s1}{\PYZsq{}}\PY{p}{)}\PY{p}{:}
            \PY{c+c1}{\PYZsh{}print(\PYZsq{}*** Importing the value of MOBILITY, DIffusion and reaction rate from a text file\PYZsq{})}
            \PY{n}{emobility}\PY{o}{=}\PY{n}{np}\PY{o}{.}\PY{n}{zeros}\PY{p}{(}\PY{n}{npoints}\PY{p}{,}\PY{n+nb}{float}\PY{p}{)}
            \PY{n}{ediffusion}\PY{o}{=}\PY{n}{np}\PY{o}{.}\PY{n}{zeros}\PY{p}{(}\PY{n}{npoints}\PY{p}{,}\PY{n+nb}{float}\PY{p}{)}
            \PY{n}{esourcee}\PY{o}{=}\PY{n}{np}\PY{o}{.}\PY{n}{zeros}\PY{p}{(}\PY{n}{npoints}\PY{p}{,}\PY{n+nb}{float}\PY{p}{)}
            \PY{n}{imobility}\PY{o}{=}\PY{n}{np}\PY{o}{.}\PY{n}{zeros}\PY{p}{(}\PY{n}{npoints}\PY{p}{,}\PY{n+nb}{float}\PY{p}{)}
            \PY{n}{idiffusion}\PY{o}{=}\PY{n}{np}\PY{o}{.}\PY{n}{zeros}\PY{p}{(}\PY{n}{npoints}\PY{p}{,}\PY{n+nb}{float}\PY{p}{)}
            \PY{n}{file}\PY{o}{=}\PY{n+nb}{open}\PY{p}{(}\PY{n}{oupfile}\PY{p}{)}
            \PY{n}{line}\PY{o}{=}\PY{n}{file}\PY{o}{.}\PY{n}{readline}\PY{p}{(}\PY{p}{)}
            \PY{k}{for} \PY{n}{data} \PY{o+ow}{in} \PY{n}{np}\PY{o}{.}\PY{n}{arange}\PY{p}{(}\PY{n}{npoints}\PY{p}{)}\PY{p}{:}
               \PY{n}{line}\PY{o}{=}\PY{n}{file}\PY{o}{.}\PY{n}{readline}\PY{p}{(}\PY{p}{)}
               \PY{n}{lineSplit}\PY{o}{=}\PY{n}{line}\PY{o}{.}\PY{n}{split}\PY{p}{(}\PY{p}{)}
               \PY{n}{emobility}\PY{p}{[}\PY{n}{data}\PY{p}{]}\PY{o}{=}\PY{n+nb}{float}\PY{p}{(}\PY{n}{lineSplit}\PY{p}{[}\PY{l+m+mi}{1}\PY{p}{]}\PY{p}{)}
            \PY{n}{line}\PY{o}{=}\PY{n}{file}\PY{o}{.}\PY{n}{readline}\PY{p}{(}\PY{p}{)}\PY{p}{;} \PY{n}{line}\PY{o}{=}\PY{n}{file}\PY{o}{.}\PY{n}{readline}\PY{p}{(}\PY{p}{)}\PY{p}{;}
            \PY{k}{for} \PY{n}{data} \PY{o+ow}{in} \PY{n}{np}\PY{o}{.}\PY{n}{arange}\PY{p}{(}\PY{n}{npoints}\PY{p}{)}\PY{p}{:}
               \PY{n}{line}\PY{o}{=}\PY{n}{file}\PY{o}{.}\PY{n}{readline}\PY{p}{(}\PY{p}{)}
               \PY{n}{lineSplit}\PY{o}{=}\PY{n}{line}\PY{o}{.}\PY{n}{split}\PY{p}{(}\PY{p}{)}
               \PY{n}{ediffusion}\PY{p}{[}\PY{n}{data}\PY{p}{]}\PY{o}{=}\PY{n+nb}{float}\PY{p}{(}\PY{n}{lineSplit}\PY{p}{[}\PY{l+m+mi}{1}\PY{p}{]}\PY{p}{)}
            \PY{n}{line}\PY{o}{=}\PY{n}{file}\PY{o}{.}\PY{n}{readline}\PY{p}{(}\PY{p}{)}\PY{p}{;} \PY{n}{line}\PY{o}{=}\PY{n}{file}\PY{o}{.}\PY{n}{readline}\PY{p}{(}\PY{p}{)}\PY{p}{;}
            \PY{k}{for} \PY{n}{data} \PY{o+ow}{in} \PY{n}{np}\PY{o}{.}\PY{n}{arange}\PY{p}{(}\PY{n}{npoints}\PY{p}{)}\PY{p}{:}
               \PY{n}{line}\PY{o}{=}\PY{n}{file}\PY{o}{.}\PY{n}{readline}\PY{p}{(}\PY{p}{)}
               \PY{n}{lineSplit}\PY{o}{=}\PY{n}{line}\PY{o}{.}\PY{n}{split}\PY{p}{(}\PY{p}{)}
               \PY{n}{esourcee}\PY{p}{[}\PY{n}{data}\PY{p}{]}\PY{o}{=}\PY{n+nb}{float}\PY{p}{(}\PY{n}{lineSplit}\PY{p}{[}\PY{l+m+mi}{1}\PY{p}{]}\PY{p}{)}
            \PY{n}{line}\PY{o}{=}\PY{n}{file}\PY{o}{.}\PY{n}{readline}\PY{p}{(}\PY{p}{)}\PY{p}{;} \PY{n}{line}\PY{o}{=}\PY{n}{file}\PY{o}{.}\PY{n}{readline}\PY{p}{(}\PY{p}{)}\PY{p}{;}
            \PY{k}{for} \PY{n}{data} \PY{o+ow}{in} \PY{n}{np}\PY{o}{.}\PY{n}{arange}\PY{p}{(}\PY{n}{npoints}\PY{p}{)}\PY{p}{:}
               \PY{n}{line}\PY{o}{=}\PY{n}{file}\PY{o}{.}\PY{n}{readline}\PY{p}{(}\PY{p}{)}
               \PY{n}{lineSplit}\PY{o}{=}\PY{n}{line}\PY{o}{.}\PY{n}{split}\PY{p}{(}\PY{p}{)}
               \PY{n}{imobility}\PY{p}{[}\PY{n}{data}\PY{p}{]}\PY{o}{=}\PY{n+nb}{float}\PY{p}{(}\PY{n}{lineSplit}\PY{p}{[}\PY{l+m+mi}{1}\PY{p}{]}\PY{p}{)}
            \PY{n}{line}\PY{o}{=}\PY{n}{file}\PY{o}{.}\PY{n}{readline}\PY{p}{(}\PY{p}{)}\PY{p}{;} \PY{n}{line}\PY{o}{=}\PY{n}{file}\PY{o}{.}\PY{n}{readline}\PY{p}{(}\PY{p}{)}\PY{p}{;}
            \PY{k}{for} \PY{n}{data} \PY{o+ow}{in} \PY{n}{np}\PY{o}{.}\PY{n}{arange}\PY{p}{(}\PY{n}{npoints}\PY{p}{)}\PY{p}{:}
               \PY{n}{line}\PY{o}{=}\PY{n}{file}\PY{o}{.}\PY{n}{readline}\PY{p}{(}\PY{p}{)}
               \PY{n}{lineSplit}\PY{o}{=}\PY{n}{line}\PY{o}{.}\PY{n}{split}\PY{p}{(}\PY{p}{)}
               \PY{n}{idiffusion}\PY{p}{[}\PY{n}{data}\PY{p}{]}\PY{o}{=}\PY{n+nb}{float}\PY{p}{(}\PY{n}{lineSplit}\PY{p}{[}\PY{l+m+mi}{1}\PY{p}{]}\PY{p}{)} 
            \PY{k}{return}\PY{p}{(}\PY{n}{emobility}\PY{p}{,}\PY{n}{ediffusion}\PY{p}{,}\PY{n}{esourcee}\PY{p}{,}\PY{n}{imobility}\PY{p}{,}\PY{n}{idiffusion}\PY{p}{)}
        \PY{n}{parameterSize}\PY{o}{=}\PY{l+m+mi}{996}  
        \PY{p}{(}\PY{n}{emobility}\PY{p}{,}\PY{n}{ediffusion}\PY{p}{,}\PY{n}{esourcee}\PY{p}{,}\PY{n}{imobility}\PY{p}{,}\PY{n}{idiffusion}\PY{p}{)} \PY{o}{=} \PY{n}{readBoltzmannParameters}\PY{p}{(}\PY{n}{parameterSize}\PY{p}{,}\PY{l+s+s1}{\PYZsq{}}\PY{l+s+s1}{output.txt}\PY{l+s+s1}{\PYZsq{}}\PY{p}{)}
\end{Verbatim}


    \paragraph{5.4. Numerical Solution of Poisson's
Equation}\label{numerical-solution-of-poissons-equation}

The Poisson's Equation is given by \[\nabla^2V=-\rho/ \epsilon\]

The finite difference form of this equation is
\[    \frac{V_{i+1}-2V_i+V_{i-1}}{dx^2}=-\rho_i/ \epsilon\]

or

\[   {V_{i+1}-2 V_{i} +V_{i-1}}=-\rho_i/ \epsilon \times {dx^2}\]

The boundary condition of the Poisson's equation is controlled by the
external voltage source. In this experiment, the bottom electrode is
grounded so the voltage on it is always O volts. Whereas the top
electrode is connected to a AC power supply so the voltage is given by
\(V=V_{peak} Sin (\omega t)\). \textbackslash{} The value of \(V_1\) to
\(V_{n-1}\) has to be determined. The set of all associated equations
involved in this process can be represented in the single matrix
equation. \[AV=B\] where, A is the tri-diagonal matrix, and V \& B are
the column matrices as given below.

    \begin{equation}
\begin{bmatrix}
    1       & 0 & 0 & 0& \dots & 0&0&0 \\
    1       & -2 & 1 &0 & \dots & 0 &0 &0\\
    0       &  1 & -2 & 1 & \dots & 0&0&0 \\
    \vdots & \vdots & \vdots&\vdots & \ddots & \vdots & \vdots & \vdots \\
    0       &  0 & 0 & 0 & \dots & 1 &-2 &1\\
    0&0&0&0&0&0&0&1
\end{bmatrix}
\begin{bmatrix}
    V_0 \\
    V_1\\
    V_2\\
    \vdots\\
    V_{n-1}\\
    V_n
\end{bmatrix}
=
-dx^2/\epsilon
\begin{bmatrix}
    V_{peak}sin(\omega t) \\
    \rho_1\\
    \rho_2\\
    \vdots\\
    \rho_{n-1}\\
    0
\end{bmatrix}
\end{equation}

    \begin{Verbatim}[commandchars=\\\{\}]
{\color{incolor}In [{\color{incolor}12}]:} \PY{c+c1}{\PYZsh{} The function to create the tridiagonal matrix used to solve the above poisson equation}
         \PY{k}{def} \PY{n+nf}{tridiagSparse}\PY{p}{(}\PY{n}{nxp}\PY{p}{,} \PY{n}{k1}\PY{o}{=}\PY{o}{\PYZhy{}}\PY{l+m+mi}{1}\PY{p}{,} \PY{n}{k2}\PY{o}{=}\PY{l+m+mi}{0}\PY{p}{,} \PY{n}{k3}\PY{o}{=}\PY{l+m+mi}{1}\PY{p}{)}\PY{p}{:}
             \PY{n}{a} \PY{o}{=} \PY{n}{np}\PY{o}{.}\PY{n}{ones}\PY{p}{(}\PY{n}{nxp}\PY{p}{)}\PY{o}{*}\PY{p}{(}\PY{o}{\PYZhy{}}\PY{l+m+mi}{1}\PY{p}{)}
             \PY{n}{b} \PY{o}{=} \PY{n}{np}\PY{o}{.}\PY{n}{ones}\PY{p}{(}\PY{n}{nxp}\PY{p}{)}\PY{o}{*}\PY{p}{(}\PY{l+m+mi}{2}\PY{p}{)}
             \PY{k}{return} \PY{n}{sparse}\PY{o}{.}\PY{n}{dia\PYZus{}matrix}\PY{p}{(}\PY{p}{(}\PY{p}{[}\PY{n}{a}\PY{p}{,}\PY{n}{b}\PY{p}{,}\PY{n}{a}\PY{p}{]}\PY{p}{,}\PY{p}{[}\PY{n}{k1}\PY{p}{,}\PY{n}{k2}\PY{p}{,}\PY{n}{k3}\PY{p}{]}\PY{p}{)}\PY{p}{,}\PY{n}{shape}\PY{o}{=}\PY{p}{(}\PY{n}{nxp}\PY{p}{,}\PY{n}{nxp}\PY{p}{)}\PY{p}{)}\PY{o}{.}\PY{n}{tocsc}\PY{p}{(}\PY{p}{)}
         
         \PY{k}{def} \PY{n+nf}{poissonmatrix}\PY{p}{(}\PY{n}{nxp}\PY{p}{,} \PY{n}{k1}\PY{o}{=}\PY{o}{\PYZhy{}}\PY{l+m+mi}{1}\PY{p}{,} \PY{n}{k2}\PY{o}{=}\PY{l+m+mi}{0}\PY{p}{,} \PY{n}{k3}\PY{o}{=}\PY{l+m+mi}{1}\PY{p}{)}\PY{p}{:}
             \PY{n}{a} \PY{o}{=} \PY{n}{np}\PY{o}{.}\PY{n}{ones}\PY{p}{(}\PY{n}{nxp}\PY{p}{)}\PY{o}{*}\PY{p}{(}\PY{l+m+mf}{2.0}\PY{p}{)}
             \PY{n}{b} \PY{o}{=} \PY{n}{np}\PY{o}{.}\PY{n}{ones}\PY{p}{(}\PY{n}{nxp}\PY{o}{\PYZhy{}}\PY{l+m+mi}{1}\PY{p}{)}\PY{o}{*}\PY{p}{(}\PY{o}{\PYZhy{}}\PY{l+m+mi}{1}\PY{p}{)}
             \PY{k}{return} \PY{n}{np}\PY{o}{.}\PY{n}{diag}\PY{p}{(}\PY{n}{b}\PY{p}{,} \PY{n}{k1}\PY{p}{)} \PY{o}{+} \PY{n}{np}\PY{o}{.}\PY{n}{diag}\PY{p}{(}\PY{n}{a}\PY{p}{,} \PY{n}{k2}\PY{p}{)} \PY{o}{+} \PY{n}{np}\PY{o}{.}\PY{n}{diag}\PY{p}{(}\PY{n}{b}\PY{p}{,} \PY{n}{k3}\PY{p}{)}
         \PY{c+c1}{\PYZsh{} How to solve? }
         \PY{c+c1}{\PYZsh{}\PYZhy{}\PYZhy{}\PYZhy{}\PYZhy{}\PYZhy{}\PYZhy{}\PYZhy{}\PYZhy{}\PYZhy{}\PYZhy{}\PYZhy{}\PYZhy{}\PYZhy{}\PYZhy{}\PYZhy{}\PYZhy{}\PYZhy{}\PYZhy{}\PYZhy{}\PYZhy{}\PYZhy{}\PYZhy{}\PYZhy{}\PYZhy{}\PYZhy{}\PYZhy{}\PYZhy{}\PYZhy{}\PYZhy{}\PYZhy{}\PYZhy{}\PYZhy{}\PYZhy{}\PYZhy{}\PYZhy{}\PYZhy{}\PYZhy{}\PYZhy{}\PYZhy{}\PYZhy{}\PYZhy{}\PYZhy{}\PYZhy{}\PYZhy{}\PYZhy{}\PYZhy{}\PYZhy{}\PYZhy{}\PYZhy{}\PYZhy{}\PYZhy{}\PYZhy{}\PYZhy{}\PYZhy{}\PYZhy{}\PYZhy{}\PYZhy{}\PYZhy{}\PYZhy{}\PYZhy{}\PYZhy{}\PYZhy{}\PYZhy{}\PYZhy{}\PYZhy{}\PYZhy{}\PYZhy{}\PYZhy{}\PYZhy{}\PYZhy{}\PYZhy{}\PYZhy{}\PYZhy{}\PYZhy{}\PYZhy{}\PYZhy{}\PYZhy{}\PYZhy{}\PYZhy{}\PYZhy{}}
         \PY{c+c1}{\PYZsh{}Maat2=tridiagSparse(ngrid\PYZhy{}2)}
         \PY{c+c1}{\PYZsh{}solvpot=la.spsolve(Maat2,chrgg)  \PYZsh{}\PYZsh{}. chrgg is a column matrix that stores the net\PYZhy{}charge at each grid points}
         \PY{c+c1}{\PYZsh{}potentl[1:\PYZhy{}1]=solvpot }
         \PY{c+c1}{\PYZsh{}\PYZhy{}\PYZhy{}\PYZhy{}\PYZhy{}\PYZhy{}\PYZhy{}\PYZhy{}\PYZhy{}\PYZhy{}\PYZhy{}\PYZhy{}\PYZhy{}\PYZhy{}\PYZhy{}\PYZhy{}\PYZhy{}\PYZhy{}\PYZhy{}\PYZhy{}\PYZhy{}\PYZhy{}\PYZhy{}\PYZhy{}\PYZhy{}\PYZhy{}\PYZhy{}\PYZhy{}\PYZhy{}\PYZhy{}\PYZhy{}\PYZhy{}\PYZhy{}\PYZhy{}\PYZhy{}\PYZhy{}\PYZhy{}\PYZhy{}\PYZhy{}\PYZhy{}\PYZhy{}\PYZhy{}\PYZhy{}\PYZhy{}\PYZhy{}\PYZhy{}\PYZhy{}\PYZhy{}\PYZhy{}\PYZhy{}\PYZhy{}\PYZhy{}\PYZhy{}\PYZhy{}\PYZhy{}\PYZhy{}\PYZhy{}\PYZhy{}\PYZhy{}\PYZhy{}\PYZhy{}\PYZhy{}\PYZhy{}\PYZhy{}\PYZhy{}\PYZhy{}\PYZhy{}\PYZhy{}\PYZhy{}\PYZhy{}\PYZhy{}\PYZhy{}\PYZhy{}\PYZhy{}\PYZhy{}\PYZhy{}\PYZhy{}\PYZhy{}\PYZhy{}\PYZhy{}\PYZhy{}}
\end{Verbatim}


    \paragraph{Function to plot the graphs and visualize the trends of
various plasma
parameters}\label{function-to-plot-the-graphs-and-visualize-the-trends-of-various-plasma-parameters}

This subprogram allows the user to pass the plasma parameters
(electron/ion densities, electric field, electric potential and
netcharge) and make 4 plots for each of them. It is called once after
each 10000 time steps.

    \begin{Verbatim}[commandchars=\\\{\}]
{\color{incolor}In [{\color{incolor}25}]:} \PY{k}{def} \PY{n+nf}{fourPlots}\PY{p}{(}\PY{n}{ones}\PY{p}{,}\PY{n}{titleone}\PY{p}{,}\PY{n}{two}\PY{p}{,}\PY{n}{titletwo}\PY{p}{,}\PY{n}{three}\PY{p}{,}\PY{n}{titlethree}\PY{p}{,}\PY{n}{four}\PY{p}{,}\PY{n}{titlefour}\PY{p}{)}\PY{p}{:}      
                 \PY{n}{f}\PY{p}{,} \PY{n}{axarr} \PY{o}{=} \PY{n}{plt}\PY{o}{.}\PY{n}{subplots}\PY{p}{(}\PY{l+m+mi}{2}\PY{p}{,} \PY{l+m+mi}{2}\PY{p}{)}
                 
                 \PY{k}{for} \PY{n}{field} \PY{o+ow}{in} \PY{n}{ones}\PY{p}{:}
                     \PY{n}{axarr}\PY{p}{[}\PY{l+m+mi}{0}\PY{p}{,}\PY{l+m+mi}{0}\PY{p}{]}\PY{o}{.}\PY{n}{plot}\PY{p}{(}\PY{n}{field}\PY{p}{)}\PY{p}{;}
                 \PY{n}{axarr}\PY{p}{[}\PY{l+m+mi}{0}\PY{p}{,}\PY{l+m+mi}{0}\PY{p}{]}\PY{o}{.}\PY{n}{set\PYZus{}title}\PY{p}{(}\PY{l+s+s2}{\PYZdq{}}\PY{l+s+s2}{Density(m\PYZhy{}3)}\PY{l+s+s2}{\PYZdq{}}\PY{p}{)}       
                 \PY{n}{axarr}\PY{p}{[}\PY{l+m+mi}{0}\PY{p}{,}\PY{l+m+mi}{1}\PY{p}{]}\PY{o}{.}\PY{n}{plot}\PY{p}{(}\PY{n}{netcharge}\PY{p}{)}
                 \PY{n}{axarr}\PY{p}{[}\PY{l+m+mi}{0}\PY{p}{,}\PY{l+m+mi}{1}\PY{p}{]}\PY{o}{.}\PY{n}{set\PYZus{}title}\PY{p}{(}\PY{l+s+s2}{\PYZdq{}}\PY{l+s+s2}{Netcharge(C)}\PY{l+s+s2}{\PYZdq{}}\PY{p}{)}
                 \PY{n}{axarr}\PY{p}{[}\PY{l+m+mi}{1}\PY{p}{,}\PY{l+m+mi}{1}\PY{p}{]}\PY{o}{.}\PY{n}{plot}\PY{p}{(}\PY{n}{potentl}\PY{p}{)}
                 \PY{n}{axarr}\PY{p}{[}\PY{l+m+mi}{1}\PY{p}{,}\PY{l+m+mi}{1}\PY{p}{]}\PY{o}{.}\PY{n}{set\PYZus{}title}\PY{p}{(}\PY{l+s+s2}{\PYZdq{}}\PY{l+s+s2}{\PYZdq{}}\PY{p}{)}       
                 \PY{n}{axarr}\PY{p}{[}\PY{l+m+mi}{1}\PY{p}{,}\PY{l+m+mi}{0}\PY{p}{]}\PY{o}{.}\PY{n}{plot}\PY{p}{(}\PY{n}{efield}\PY{p}{)}      
                 \PY{n}{axarr}\PY{p}{[}\PY{l+m+mi}{1}\PY{p}{,}\PY{l+m+mi}{0}\PY{p}{]}\PY{o}{.}\PY{n}{set\PYZus{}title}\PY{p}{(}\PY{l+s+s2}{\PYZdq{}}\PY{l+s+s2}{\PYZdq{}}\PY{p}{)}
                 \PY{n}{f}\PY{o}{.}\PY{n}{subplots\PYZus{}adjust}\PY{p}{(}\PY{n}{hspace}\PY{o}{=}\PY{l+m+mf}{0.3}\PY{p}{)}
                 \PY{n}{plt}\PY{o}{.}\PY{n}{show}\PY{p}{(}\PY{p}{)}
\end{Verbatim}


    \paragraph{5.5. Numerical Solution of Continuity
Equation}\label{numerical-solution-of-continuity-equation}

The continuity equation is solved using the implicit solver. The
boundary condition for the advection component is different from that of
the diffusion component.

    \begin{Verbatim}[commandchars=\\\{\}]
{\color{incolor}In [{\color{incolor}10}]:} \PY{k}{def} \PY{n+nf}{SparseContinuityOperator}\PY{p}{(}\PY{n}{dif}\PY{p}{,}\PY{n}{dx}\PY{p}{,}\PY{n}{dt}\PY{p}{,}\PY{n}{vi}\PY{p}{,}\PY{n}{k1}\PY{o}{=}\PY{o}{\PYZhy{}}\PY{l+m+mi}{1}\PY{p}{,}\PY{n}{k2}\PY{o}{=}\PY{l+m+mi}{0}\PY{p}{,}\PY{n}{k3}\PY{o}{=}\PY{l+m+mi}{1}\PY{p}{)}\PY{p}{:}
             \PY{n}{nx}\PY{o}{=}\PY{n}{dif}\PY{o}{.}\PY{n}{size}
             \PY{n}{d1}\PY{o}{=}\PY{n}{np}\PY{o}{.}\PY{n}{zeros}\PY{p}{(}\PY{p}{(}\PY{n}{nx}\PY{p}{)}\PY{p}{,}\PY{n+nb}{float}\PY{p}{)}
             \PY{n}{d2}\PY{o}{=}\PY{n}{np}\PY{o}{.}\PY{n}{ones}\PY{p}{(}\PY{p}{(}\PY{n}{nx}\PY{p}{)}\PY{p}{,}\PY{n+nb}{float}\PY{p}{)}
             \PY{n}{d3}\PY{o}{=}\PY{n}{np}\PY{o}{.}\PY{n}{zeros}\PY{p}{(}\PY{p}{(}\PY{n}{nx}\PY{p}{)}\PY{p}{,}\PY{n+nb}{float}\PY{p}{)}
             \PY{n}{graddif}\PY{o}{=}\PY{p}{(}\PY{n}{dif}\PY{p}{[}\PY{l+m+mi}{2}\PY{p}{:}\PY{p}{]}\PY{o}{\PYZhy{}}\PY{n}{dif}\PY{p}{[}\PY{p}{:}\PY{o}{\PYZhy{}}\PY{l+m+mi}{2}\PY{p}{]}\PY{p}{)}\PY{o}{/}\PY{p}{(}\PY{l+m+mi}{2}\PY{o}{*}\PY{n}{dx}\PY{p}{)}
             \PY{n}{avM}\PY{o}{=}\PY{l+m+mf}{0.5}\PY{o}{*}\PY{p}{(}\PY{n}{vi}\PY{p}{[}\PY{p}{:}\PY{o}{\PYZhy{}}\PY{l+m+mi}{2}\PY{p}{]}\PY{o}{+}\PY{n}{vi}\PY{p}{[}\PY{l+m+mi}{1}\PY{p}{:}\PY{o}{\PYZhy{}}\PY{l+m+mi}{1}\PY{p}{]}\PY{p}{)}
             \PY{n}{avP}\PY{o}{=}\PY{l+m+mf}{0.5}\PY{o}{*}\PY{p}{(}\PY{n}{vi}\PY{p}{[}\PY{l+m+mi}{1}\PY{p}{:}\PY{o}{\PYZhy{}}\PY{l+m+mi}{1}\PY{p}{]}\PY{o}{+}\PY{n}{vi}\PY{p}{[}\PY{l+m+mi}{2}\PY{p}{:}\PY{p}{]}\PY{p}{)}
             \PY{n}{d1}\PY{p}{[}\PY{p}{:}\PY{o}{\PYZhy{}}\PY{l+m+mi}{2}\PY{p}{]}\PY{o}{=}\PY{o}{\PYZhy{}}\PY{n}{dt}\PY{o}{*}\PY{n}{dif}\PY{p}{[}\PY{l+m+mi}{1}\PY{p}{:}\PY{o}{\PYZhy{}}\PY{l+m+mi}{1}\PY{p}{]}\PY{o}{/}\PY{p}{(}\PY{n}{dx}\PY{o}{*}\PY{o}{*}\PY{l+m+mi}{2}\PY{p}{)}\PY{o}{+}\PY{n}{dt}\PY{o}{*}\PY{n}{graddif}\PY{o}{/}\PY{p}{(}\PY{l+m+mi}{2}\PY{o}{*}\PY{n}{dx}\PY{p}{)}\PY{o}{+}\PY{n}{dt}\PY{o}{*}\PY{n}{avM}\PY{o}{/}\PY{p}{(}\PY{l+m+mi}{2}\PY{o}{*}\PY{n}{dx}\PY{p}{)}\PY{o}{+}\PY{n}{dt}\PY{o}{*}\PY{n}{vi}\PY{p}{[}\PY{l+m+mi}{0}\PY{p}{:}\PY{o}{\PYZhy{}}\PY{l+m+mi}{2}\PY{p}{]}\PY{o}{/}\PY{p}{(}\PY{l+m+mi}{2}\PY{o}{*}\PY{n}{dx}\PY{p}{)}
             \PY{n}{d2}\PY{p}{[}\PY{l+m+mi}{1}\PY{p}{:}\PY{o}{\PYZhy{}}\PY{l+m+mi}{1}\PY{p}{]}\PY{o}{=}\PY{p}{(}\PY{l+m+mi}{1}\PY{o}{+}\PY{l+m+mi}{2}\PY{o}{*}\PY{n}{dt}\PY{o}{*}\PY{n}{dif}\PY{p}{[}\PY{l+m+mi}{1}\PY{p}{:}\PY{o}{\PYZhy{}}\PY{l+m+mi}{1}\PY{p}{]}\PY{o}{/}\PY{p}{(}\PY{n}{dx}\PY{o}{*}\PY{o}{*}\PY{l+m+mi}{2}\PY{p}{)}\PY{p}{)}\PY{o}{+}\PY{n}{dt}\PY{o}{*}\PY{p}{(}\PY{o}{\PYZhy{}}\PY{n}{avM}\PY{o}{/}\PY{p}{(}\PY{l+m+mi}{2}\PY{o}{*}\PY{n}{dx}\PY{p}{)}\PY{o}{+}\PY{n}{vi}\PY{p}{[}\PY{l+m+mi}{1}\PY{p}{:}\PY{o}{\PYZhy{}}\PY{l+m+mi}{1}\PY{p}{]}\PY{o}{/}\PY{p}{(}\PY{l+m+mi}{2}\PY{o}{*}\PY{n}{dx}\PY{p}{)}\PY{o}{\PYZhy{}}\PY{n}{avP}\PY{o}{/}\PY{p}{(}\PY{l+m+mi}{2}\PY{o}{*}\PY{n}{dx}\PY{p}{)}\PY{o}{\PYZhy{}}\PY{n}{vi}\PY{p}{[}\PY{l+m+mi}{1}\PY{p}{:}\PY{o}{\PYZhy{}}\PY{l+m+mi}{1}\PY{p}{]}\PY{o}{/}\PY{p}{(}\PY{l+m+mi}{2}\PY{o}{*}\PY{n}{dx}\PY{p}{)}\PY{p}{)}
             \PY{n}{d3}\PY{p}{[}\PY{l+m+mi}{2}\PY{p}{:}\PY{p}{]}\PY{o}{=}\PY{o}{\PYZhy{}}\PY{p}{(}\PY{n}{dt}\PY{o}{*}\PY{n}{graddif}\PY{o}{/}\PY{p}{(}\PY{l+m+mi}{2}\PY{o}{*}\PY{n}{dx}\PY{p}{)}\PY{o}{+}\PY{n}{dt}\PY{o}{*}\PY{n}{dif}\PY{p}{[}\PY{l+m+mi}{1}\PY{p}{:}\PY{o}{\PYZhy{}}\PY{l+m+mi}{1}\PY{p}{]}\PY{o}{/}\PY{p}{(}\PY{n}{dx}\PY{o}{*}\PY{o}{*}\PY{l+m+mi}{2}\PY{p}{)}\PY{p}{)}\PY{o}{+}\PY{n}{dt}\PY{o}{*}\PY{p}{(}\PY{n}{avP}\PY{o}{/}\PY{p}{(}\PY{l+m+mi}{2}\PY{o}{*}\PY{n}{dx}\PY{p}{)}\PY{o}{\PYZhy{}}\PY{n}{vi}\PY{p}{[}\PY{l+m+mi}{2}\PY{p}{:}\PY{p}{]}\PY{o}{/}\PY{p}{(}\PY{l+m+mi}{2}\PY{o}{*}\PY{n}{dx}\PY{p}{)}\PY{p}{)}
             \PY{k}{return} \PY{p}{(}\PY{n}{sparse}\PY{o}{.}\PY{n}{dia\PYZus{}matrix}\PY{p}{(}\PY{p}{(}\PY{p}{[}\PY{n}{d1}\PY{p}{,}\PY{n}{d2}\PY{p}{,}\PY{n}{d3}\PY{p}{]}\PY{p}{,}\PY{p}{[}\PY{n}{k1}\PY{p}{,}\PY{n}{k2}\PY{p}{,}\PY{n}{k3}\PY{p}{]}\PY{p}{)}\PY{p}{,}\PY{n}{shape}\PY{o}{=}\PY{p}{(}\PY{n}{nx}\PY{p}{,}\PY{n}{nx}\PY{p}{)}\PY{p}{)}\PY{o}{.}\PY{n}{tocsc}\PY{p}{(}\PY{p}{)} \PY{p}{)}
\end{Verbatim}


    \begin{Verbatim}[commandchars=\\\{\}]
{\color{incolor}In [{\color{incolor}26}]:} \PY{c+c1}{\PYZsh{}=======================time Loop======================================================================}
         \PY{c+c1}{\PYZsh{}storeResults=np.zeros((numberOfSteps,5,ngrid),float)}
         \PY{n}{Maat1}\PY{o}{=}\PY{n}{poissonmatrix}\PY{p}{(}\PY{n}{ngrid}\PY{o}{\PYZhy{}}\PY{l+m+mi}{2}\PY{p}{)}
         \PY{n}{Maat2}\PY{o}{=}\PY{n}{tridiagSparse}\PY{p}{(}\PY{n}{ngrid}\PY{o}{\PYZhy{}}\PY{l+m+mi}{2}\PY{p}{)}
         \PY{n}{invertedmat}\PY{o}{=}\PY{n}{la}\PY{o}{.}\PY{n}{inv}\PY{p}{(}\PY{n}{Maat2}\PY{p}{)}
         \PY{n}{numberOfSteps} \PY{o}{=} \PY{l+m+mi}{1000000000}
         
         \PY{c+c1}{\PYZsh{}ndensity = np.ones((2,ngrid0+2),float)*1000.0}
         \PY{n}{ndensity}\PY{o}{=}\PY{l+m+mi}{1000}\PY{o}{*}\PY{n}{np}\PY{o}{.}\PY{n}{random}\PY{o}{.}\PY{n}{rand}\PY{p}{(}\PY{l+m+mi}{2}\PY{p}{,}\PY{n}{ngrid0}\PY{o}{+}\PY{l+m+mi}{2}\PY{p}{)}
         
         \PY{n}{oo}\PY{o}{=}\PY{l+m+mi}{1}
         \PY{n}{tymestep2}\PY{o}{=}\PY{l+m+mf}{0.0}
         \PY{n}{tyme}\PY{o}{=}\PY{l+m+mf}{0.0}
         
         \PY{k}{for} \PY{n}{tymeStep} \PY{o+ow}{in} \PY{n+nb}{range}\PY{p}{(}\PY{l+m+mi}{1}\PY{p}{,}\PY{l+m+mi}{2001}\PY{p}{)}\PY{p}{:}
             \PY{n}{tyme}\PY{o}{=}\PY{n}{tyme}\PY{o}{+}\PY{n}{dt}
            \PY{c+c1}{\PYZsh{}poission equation}
             \PY{c+c1}{\PYZsh{}==================================================================================================================}
             \PY{n}{netcharge}\PY{p}{[}\PY{p}{:}\PY{p}{]}\PY{o}{=}\PY{l+m+mf}{0.0} \PY{c+c1}{\PYZsh{}clear the garbage value from pervious loop}
             \PY{k}{for} \PY{n}{i} \PY{o+ow}{in} \PY{n}{np}\PY{o}{.}\PY{n}{arange}\PY{p}{(}\PY{n}{ns}\PY{p}{)}\PY{p}{:}
                 \PY{n}{netcharge}\PY{p}{[}\PY{n}{nwd1}\PY{p}{:}\PY{n}{nwd1}\PY{o}{+}\PY{l+m+mi}{2}\PY{o}{+}\PY{n}{ngrid0}\PY{p}{]}\PY{o}{+}\PY{o}{=}\PY{n}{ee}\PY{o}{*}\PY{n}{ncharge}\PY{p}{[}\PY{n}{i}\PY{p}{]}\PY{o}{*}\PY{n}{ndensity}\PY{p}{[}\PY{n}{i}\PY{p}{,}\PY{p}{:}\PY{p}{]}  \PY{c+c1}{\PYZsh{}calculating the net charge at each grid points}
             \PY{n}{leftPot}\PY{o}{=}\PY{l+m+mf}{1.0}\PY{o}{*}\PY{n}{volt}\PY{o}{*}\PY{n}{np}\PY{o}{.}\PY{n}{sin}\PY{p}{(}\PY{l+m+mi}{2}\PY{o}{*}\PY{n}{np}\PY{o}{.}\PY{n}{pi}\PY{o}{*}\PY{n}{tyme}\PY{o}{*}\PY{n}{frequencySource}\PY{p}{)} \PY{c+c1}{\PYZsh{} frequency of the source is given by a parameter}
             \PY{n}{rightpot}\PY{o}{=}\PY{l+m+mf}{0.0}\PY{o}{*}\PY{n}{volt}\PY{o}{*}\PY{n}{np}\PY{o}{.}\PY{n}{sin}\PY{p}{(}\PY{l+m+mi}{2}\PY{o}{*}\PY{n}{np}\PY{o}{.}\PY{n}{pi}\PY{o}{*}\PY{n}{tyme}\PY{o}{*}\PY{n}{frequencySource}\PY{p}{)}
             \PY{n}{chrgg}\PY{o}{=}\PY{p}{(}\PY{n}{netcharge}\PY{p}{[}\PY{l+m+mi}{1}\PY{p}{:}\PY{o}{\PYZhy{}}\PY{l+m+mi}{1}\PY{p}{]}\PY{o}{/}\PY{n}{e0}\PY{p}{)}\PY{o}{*}\PY{n}{dx}\PY{o}{*}\PY{n}{dx}
             \PY{n}{chrgg}\PY{p}{[}\PY{l+m+mi}{0}\PY{p}{]}\PY{o}{=}\PY{n}{chrgg}\PY{p}{[}\PY{l+m+mi}{0}\PY{p}{]}\PY{o}{+}\PY{n}{leftPot}
             \PY{n}{chrgg}\PY{p}{[}\PY{o}{\PYZhy{}}\PY{l+m+mi}{1}\PY{p}{]}\PY{o}{=}\PY{n}{chrgg}\PY{p}{[}\PY{o}{\PYZhy{}}\PY{l+m+mi}{1}\PY{p}{]}\PY{o}{+}\PY{n}{rightpot}
             \PY{n}{potentl}\PY{p}{[}\PY{l+m+mi}{0}\PY{p}{]}\PY{o}{=}\PY{n}{leftPot}
             \PY{n}{potentl}\PY{p}{[}\PY{o}{\PYZhy{}}\PY{l+m+mi}{1}\PY{p}{]}\PY{o}{=}\PY{n}{rightpot}
             \PY{n}{solvpot}\PY{o}{=}\PY{n}{la}\PY{o}{.}\PY{n}{spsolve}\PY{p}{(}\PY{n}{Maat2}\PY{p}{,}\PY{n}{chrgg}\PY{p}{)}
             \PY{n}{potentl}\PY{p}{[}\PY{l+m+mi}{1}\PY{p}{:}\PY{o}{\PYZhy{}}\PY{l+m+mi}{1}\PY{p}{]}\PY{o}{=}\PY{n}{solvpot} 
             
             \PY{c+c1}{\PYZsh{}**calculate electric field as negative gradient of potential (Expressed in Townsend Unit)}
             \PY{n}{efield}\PY{p}{[}\PY{p}{:}\PY{p}{]}\PY{o}{=}\PY{o}{\PYZhy{}}\PY{n}{townsendunit}\PY{o}{*}\PY{p}{(}\PY{n}{potentl}\PY{p}{[}\PY{n}{nwd1}\PY{o}{+}\PY{l+m+mi}{1}\PY{p}{:}\PY{n}{nwd1}\PY{o}{+}\PY{l+m+mi}{3}\PY{o}{+}\PY{n}{ngrid0}\PY{p}{]}\PY{o}{\PYZhy{}}\PY{n}{potentl}\PY{p}{[}\PY{n}{nwd1}\PY{o}{\PYZhy{}}\PY{l+m+mi}{1}\PY{p}{:}\PY{n}{nwd1}\PY{o}{+}\PY{l+m+mi}{1}\PY{o}{+}\PY{n}{ngrid0}\PY{p}{]}\PY{p}{)}\PY{o}{/}\PY{p}{(}\PY{l+m+mf}{2.0}\PY{o}{*}\PY{n}{dx}\PY{p}{)}
             \PY{n}{efield}\PY{p}{[}\PY{l+m+mi}{0}\PY{p}{]}\PY{o}{=}\PY{o}{\PYZhy{}}\PY{n}{townsendunit}\PY{o}{*}\PY{p}{(}\PY{o}{\PYZhy{}}\PY{p}{(}\PY{l+m+mf}{11.0}\PY{o}{/}\PY{l+m+mi}{6}\PY{p}{)}\PY{o}{*}\PY{n}{potentl}\PY{p}{[}\PY{n}{nwd1}\PY{p}{]}\PY{o}{+}\PY{l+m+mf}{3.0}\PY{o}{*}\PY{n}{potentl}\PY{p}{[}\PY{n}{nwd1}\PY{o}{+}\PY{l+m+mi}{1}\PY{p}{]}\PY{o}{\PYZhy{}}\PY{p}{(}\PY{l+m+mf}{3.0}\PY{o}{/}\PY{l+m+mi}{2}\PY{p}{)}\PY{o}{*}\PY{n}{potentl}\PY{p}{[}\PY{n}{nwd1}\PY{o}{+}\PY{l+m+mi}{2}\PY{p}{]}\PY{o}{+}\PY{p}{(}\PY{l+m+mf}{1.0}\PY{o}{/}\PY{l+m+mi}{3}\PY{p}{)}\PY{o}{*}\PY{n}{potentl}\PY{p}{[}\PY{n}{nwd1}\PY{o}{+}\PY{l+m+mi}{3}\PY{p}{]}\PY{p}{)}\PY{o}{/}\PY{n}{dx}
             \PY{n}{efield}\PY{p}{[}\PY{o}{\PYZhy{}}\PY{l+m+mi}{1}\PY{p}{]}\PY{o}{=}\PY{o}{\PYZhy{}}\PY{n}{townsendunit}\PY{o}{*}\PY{p}{(}\PY{n}{potentl}\PY{p}{[}\PY{n}{nwd1}\PY{o}{+}\PY{l+m+mi}{1}\PY{o}{+}\PY{n}{ngrid0}\PY{p}{]}\PY{o}{\PYZhy{}}\PY{n}{potentl}\PY{p}{[}\PY{n}{nwd1}\PY{o}{+}\PY{n}{ngrid0}\PY{p}{]}\PY{p}{)}\PY{o}{/}\PY{n}{dx}
            
             \PY{k}{if} \PY{n+nb}{any}\PY{p}{(}\PY{n+nb}{abs}\PY{p}{(}\PY{n}{efield}\PY{p}{[}\PY{p}{:}\PY{p}{]}\PY{p}{)}\PY{o}{\PYZgt{}}\PY{l+m+mi}{1000}\PY{p}{)}\PY{p}{:}\PY{c+c1}{\PYZsh{}All the reaction coefficients are calculated for efield\PYZlt{}npoints. Value more than that will imply that the there is something wrong in the simulation}
                \PY{n}{f}\PY{o}{=} \PY{n+nb}{open}\PY{p}{(}\PY{l+s+s2}{\PYZdq{}}\PY{l+s+s2}{logfile.txt}\PY{l+s+s2}{\PYZdq{}}\PY{p}{,}\PY{l+s+s2}{\PYZdq{}}\PY{l+s+s2}{w+}\PY{l+s+s2}{\PYZdq{}}\PY{p}{)}
                \PY{n}{f}\PY{o}{.}\PY{n}{write}\PY{p}{(}\PY{l+s+s2}{\PYZdq{}}\PY{l+s+s2}{Error!! The value of Electric field exceeded limit. Something might be wrong!!}\PY{l+s+s2}{\PYZdq{}}\PY{p}{)}
                \PY{n}{sys}\PY{o}{.}\PY{n}{exit}\PY{p}{(}\PY{p}{)}
                
             \PY{c+c1}{\PYZsh{}calculating the coefficients (Interpolation..)\PYZhy{}\PYZhy{}\PYZhy{}\PYZhy{}\PYZhy{}\PYZhy{}\PYZhy{}\PYZhy{}\PYZhy{}\PYZhy{}\PYZhy{}\PYZhy{}\PYZhy{}\PYZhy{}\PYZhy{}\PYZhy{}\PYZhy{}\PYZhy{}\PYZhy{}\PYZhy{}\PYZhy{}\PYZhy{}\PYZhy{}\PYZhy{}\PYZhy{}\PYZhy{}\PYZhy{}\PYZhy{}\PYZhy{}\PYZhy{}\PYZhy{}\PYZhy{}\PYZhy{}\PYZhy{}\PYZhy{}\PYZhy{}\PYZhy{}\PYZhy{}\PYZhy{}\PYZhy{}\PYZhy{}\PYZhy{}\PYZhy{}\PYZhy{}\PYZhy{}\PYZhy{}\PYZhy{}\PYZhy{}\PYZhy{}\PYZhy{}\PYZhy{}\PYZhy{}\PYZhy{}\PYZhy{}\PYZhy{}\PYZhy{}\PYZhy{}\PYZhy{}\PYZhy{}\PYZhy{}\PYZhy{}\PYZhy{}\PYZhy{}\PYZhy{}\PYZhy{}\PYZhy{}\PYZhy{}\PYZhy{}\PYZhy{}\PYZhy{}\PYZhy{}\PYZhy{}\PYZhy{}\PYZhy{}\PYZhy{}\PYZhy{}\PYZhy{}\PYZhy{}\PYZhy{}\PYZhy{}\PYZhy{}\PYZhy{}\PYZhy{}\PYZhy{}\PYZhy{}}
             \PY{n}{indlocate}\PY{o}{=}\PY{n+nb}{abs}\PY{p}{(}\PY{n}{efield}\PY{p}{[}\PY{p}{:}\PY{p}{]}\PY{p}{)}\PY{o}{.}\PY{n}{astype}\PY{p}{(}\PY{n+nb}{int}\PY{p}{)}
             \PY{n}{mobegrid}\PY{o}{=}\PY{o}{\PYZhy{}}\PY{l+m+mf}{1.0}\PY{o}{*}\PY{p}{(}\PY{n}{emobility}\PY{p}{[}\PY{n}{indlocate}\PY{p}{]}\PY{o}{+}\PY{p}{(}\PY{n}{emobility}\PY{p}{[}\PY{n}{indlocate}\PY{o}{+}\PY{l+m+mi}{1}\PY{p}{]}\PY{o}{\PYZhy{}}\PY{n}{emobility}\PY{p}{[}\PY{n}{indlocate}\PY{p}{]}\PY{p}{)}\PY{o}{*}\PY{p}{(}\PY{n+nb}{abs}\PY{p}{(}\PY{n}{efield}\PY{p}{)}\PY{o}{\PYZhy{}}\PY{n}{indlocate}\PY{p}{)}\PY{p}{)}\PY{o}{/}\PY{n}{gasdens}
             \PY{n}{difegrid}\PY{o}{=}\PY{l+m+mf}{1.0}\PY{o}{*}\PY{p}{(}\PY{n}{ediffusion}\PY{p}{[}\PY{n}{indlocate}\PY{p}{]}\PY{o}{+}\PY{p}{(}\PY{p}{(}\PY{n}{ediffusion}\PY{p}{[}\PY{n}{indlocate}\PY{o}{+}\PY{l+m+mi}{1}\PY{p}{]}\PY{o}{\PYZhy{}}\PY{n}{ediffusion}\PY{p}{[}\PY{n}{indlocate}\PY{p}{]}\PY{p}{)}\PY{o}{*}\PY{p}{(}\PY{n+nb}{abs}\PY{p}{(}\PY{n}{efield}\PY{p}{)}\PY{o}{\PYZhy{}}\PY{n}{indlocate}\PY{p}{)}\PY{p}{)}\PY{p}{)}\PY{o}{/}\PY{n}{gasdens}
             \PY{n}{sourceegrid}\PY{o}{=}\PY{l+m+mf}{1.0}\PY{o}{*}\PY{p}{(}\PY{n}{esourcee}\PY{p}{[}\PY{n}{indlocate}\PY{p}{]}\PY{o}{+}\PY{p}{(}\PY{n}{esourcee}\PY{p}{[}\PY{n}{indlocate}\PY{o}{+}\PY{l+m+mi}{1}\PY{p}{]}\PY{o}{\PYZhy{}}\PY{n}{esourcee}\PY{p}{[}\PY{n}{indlocate}\PY{p}{]}\PY{p}{)}\PY{o}{*}\PY{p}{(}\PY{n+nb}{abs}\PY{p}{(}\PY{n}{efield}\PY{p}{)}\PY{o}{\PYZhy{}}\PY{n}{indlocate}\PY{p}{)}\PY{p}{)}\PY{o}{*}\PY{n}{gasdens}
             \PY{n}{mobigrid}\PY{o}{=}\PY{l+m+mf}{1.0}\PY{o}{*}\PY{p}{(}\PY{n}{imobility}\PY{p}{[}\PY{n}{indlocate}\PY{p}{]}\PY{o}{+}\PY{p}{(}\PY{n}{imobility}\PY{p}{[}\PY{n}{indlocate}\PY{o}{+}\PY{l+m+mi}{1}\PY{p}{]}\PY{o}{\PYZhy{}}\PY{n}{imobility}\PY{p}{[}\PY{n}{indlocate}\PY{p}{]}\PY{p}{)}\PY{o}{*}\PY{p}{(}\PY{n+nb}{abs}\PY{p}{(}\PY{n}{efield}\PY{p}{)}\PY{o}{\PYZhy{}}\PY{n}{indlocate}\PY{p}{)}\PY{p}{)}
             \PY{n}{difigrid}\PY{o}{=}\PY{l+m+mf}{1.0}\PY{o}{*}\PY{n}{idiffusion}\PY{p}{[}\PY{n}{indlocate}\PY{p}{]}\PY{o}{+}\PY{p}{(}\PY{p}{(}\PY{n}{idiffusion}\PY{p}{[}\PY{n}{indlocate}\PY{o}{+}\PY{l+m+mi}{1}\PY{p}{]}\PY{o}{\PYZhy{}}\PY{n}{idiffusion}\PY{p}{[}\PY{n}{indlocate}\PY{p}{]}\PY{p}{)}\PY{o}{*}\PY{p}{(}\PY{n+nb}{abs}\PY{p}{(}\PY{n}{efield}\PY{p}{)}\PY{o}{\PYZhy{}}\PY{n}{indlocate}\PY{p}{)}\PY{p}{)}
             
            
            \PY{c+c1}{\PYZsh{}Advection and Diffusion\PYZhy{}\PYZhy{}\PYZhy{}\PYZhy{}\PYZhy{}\PYZhy{}\PYZhy{}\PYZhy{}\PYZhy{}\PYZhy{}\PYZhy{}\PYZhy{}\PYZhy{}\PYZhy{}\PYZhy{}\PYZhy{}\PYZhy{}\PYZhy{}\PYZhy{}\PYZhy{}\PYZhy{}\PYZhy{}\PYZhy{}\PYZhy{}\PYZhy{}\PYZhy{}\PYZhy{}\PYZhy{}\PYZhy{}\PYZhy{}\PYZhy{}\PYZhy{}\PYZhy{}\PYZhy{}\PYZhy{}\PYZhy{}\PYZhy{}\PYZhy{}\PYZhy{}\PYZhy{}\PYZhy{}\PYZhy{}\PYZhy{}\PYZhy{}\PYZhy{}\PYZhy{}\PYZhy{}\PYZhy{}\PYZhy{}\PYZhy{}\PYZhy{}\PYZhy{}\PYZhy{}\PYZhy{}\PYZhy{}}
            \PY{c+c1}{\PYZsh{}==============================================================================}
             \PY{n}{ndentemp}\PY{o}{=}\PY{n}{np}\PY{o}{.}\PY{n}{zeros}\PY{p}{(}\PY{p}{(}\PY{n}{ns}\PY{p}{,}\PY{n}{ngrid0}\PY{o}{+}\PY{l+m+mi}{2}\PY{p}{)}\PY{p}{,}\PY{n+nb}{float}\PY{p}{)}
             \PY{n}{ndentemp}\PY{p}{[}\PY{l+m+mi}{0}\PY{p}{]}\PY{o}{=}\PY{n}{ndensity}\PY{p}{[}\PY{l+m+mi}{0}\PY{p}{,}\PY{p}{:}\PY{p}{]}\PY{o}{.}\PY{n}{copy}\PY{p}{(}\PY{p}{)}
             \PY{n}{ndentemp}\PY{p}{[}\PY{l+m+mi}{1}\PY{p}{]}\PY{o}{=}\PY{n}{ndensity}\PY{p}{[}\PY{l+m+mi}{1}\PY{p}{,}\PY{p}{:}\PY{p}{]}\PY{o}{.}\PY{n}{copy}\PY{p}{(}\PY{p}{)}   
             \PY{n}{ndentemp}\PY{p}{[}\PY{l+m+mi}{0}\PY{p}{,}\PY{l+m+mi}{0}\PY{p}{]}\PY{o}{=}\PY{n}{ndentemp}\PY{p}{[}\PY{l+m+mi}{0}\PY{p}{,}\PY{l+m+mi}{0}\PY{p}{]}\PY{o}{\PYZhy{}}\PY{p}{(}\PY{n}{sig\PYZus{}e\PYZus{}left}\PY{p}{)}\PY{o}{/}\PY{n}{dx}\PY{p}{;} \PY{n}{ndentemp}\PY{p}{[}\PY{l+m+mi}{0}\PY{p}{,}\PY{o}{\PYZhy{}}\PY{l+m+mi}{1}\PY{p}{]}\PY{o}{=}\PY{n}{ndentemp}\PY{p}{[}\PY{l+m+mi}{0}\PY{p}{,}\PY{o}{\PYZhy{}}\PY{l+m+mi}{1}\PY{p}{]}\PY{o}{\PYZhy{}}\PY{p}{(}\PY{n}{sig\PYZus{}e\PYZus{}right}\PY{p}{)}\PY{o}{/}\PY{n}{dx}  \PY{p}{;}\PY{n}{ndentemp}\PY{p}{[}\PY{l+m+mi}{1}\PY{p}{,}\PY{l+m+mi}{0}\PY{p}{]}\PY{o}{=}\PY{n}{ndentemp}\PY{p}{[}\PY{l+m+mi}{1}\PY{p}{,}\PY{l+m+mi}{0}\PY{p}{]}\PY{o}{\PYZhy{}}\PY{p}{(}\PY{n}{sig\PYZus{}i\PYZus{}left}\PY{p}{)}\PY{o}{/}\PY{n}{dx}\PY{p}{;}\PY{n}{ndentemp}\PY{p}{[}\PY{l+m+mi}{1}\PY{p}{,}\PY{o}{\PYZhy{}}\PY{l+m+mi}{1}\PY{p}{]}\PY{o}{=}\PY{n}{ndentemp}\PY{p}{[}\PY{l+m+mi}{1}\PY{p}{,}\PY{o}{\PYZhy{}}\PY{l+m+mi}{1}\PY{p}{]}\PY{o}{\PYZhy{}}\PY{p}{(}\PY{n}{sig\PYZus{}i\PYZus{}right}\PY{p}{)}\PY{o}{/}\PY{n}{dx} \PY{c+c1}{\PYZsh{}mirror boundary condition}
             
             \PY{n}{difOperatorElectron}\PY{o}{=}\PY{n}{SparseContinuityOperator}\PY{p}{(}\PY{n}{difegrid}\PY{p}{,}\PY{n}{dx}\PY{p}{,}\PY{n}{dt}\PY{p}{,}\PY{l+m+mi}{0}\PY{o}{*}\PY{n}{efield}\PY{o}{*}\PY{n}{mobegrid}\PY{p}{)}
             \PY{n}{difOperatorIon}\PY{o}{=}\PY{n}{SparseContinuityOperator}\PY{p}{(}\PY{n}{difigrid}\PY{p}{,}\PY{n}{dx}\PY{p}{,}\PY{n}{dt}\PY{p}{,}\PY{l+m+mi}{0}\PY{o}{*}\PY{n}{efield}\PY{o}{*}\PY{n}{mobigrid}\PY{p}{)}
             \PY{n}{ndentemp}\PY{p}{[}\PY{l+m+mi}{0}\PY{p}{,}\PY{l+m+mi}{1}\PY{p}{:}\PY{o}{\PYZhy{}}\PY{l+m+mi}{1}\PY{p}{]}\PY{o}{=}\PY{n}{la}\PY{o}{.}\PY{n}{spsolve}\PY{p}{(}\PY{n}{difOperatorElectron}\PY{p}{,}\PY{n}{ndentemp}\PY{p}{[}\PY{l+m+mi}{0}\PY{p}{]}\PY{p}{)}\PY{p}{[}\PY{l+m+mi}{1}\PY{p}{:}\PY{o}{\PYZhy{}}\PY{l+m+mi}{1}\PY{p}{]}\PY{o}{.}\PY{n}{copy}\PY{p}{(}\PY{p}{)}
             \PY{n}{ndentemp}\PY{p}{[}\PY{l+m+mi}{1}\PY{p}{,}\PY{l+m+mi}{1}\PY{p}{:}\PY{o}{\PYZhy{}}\PY{l+m+mi}{1}\PY{p}{]}\PY{o}{=}\PY{n}{la}\PY{o}{.}\PY{n}{spsolve}\PY{p}{(}\PY{n}{difOperatorIon}\PY{p}{,}\PY{n}{ndentemp}\PY{p}{[}\PY{l+m+mi}{1}\PY{p}{]}\PY{p}{)}\PY{p}{[}\PY{l+m+mi}{1}\PY{p}{:}\PY{o}{\PYZhy{}}\PY{l+m+mi}{1}\PY{p}{]}\PY{o}{.}\PY{n}{copy}\PY{p}{(}\PY{p}{)}
             \PY{n}{ndentemp}\PY{p}{[}\PY{l+m+mi}{0}\PY{p}{,}\PY{l+m+mi}{0}\PY{p}{]}\PY{o}{=}\PY{l+m+mf}{0.}\PY{p}{;} \PY{n}{ndentemp}\PY{p}{[}\PY{l+m+mi}{0}\PY{p}{,}\PY{o}{\PYZhy{}}\PY{l+m+mi}{1}\PY{p}{]}\PY{o}{=}\PY{l+m+mf}{0.}  \PY{p}{;}\PY{n}{ndentemp}\PY{p}{[}\PY{l+m+mi}{1}\PY{p}{,}\PY{l+m+mi}{0}\PY{p}{]}\PY{o}{=}\PY{l+m+mf}{0.}\PY{p}{;}\PY{n}{ndentemp}\PY{p}{[}\PY{l+m+mi}{1}\PY{p}{,}\PY{o}{\PYZhy{}}\PY{l+m+mi}{1}\PY{p}{]}\PY{o}{=}\PY{l+m+mf}{0.} \PY{c+c1}{\PYZsh{}mirror boundary condition}
             
             
             \PY{n}{difOperatorElectron1}\PY{o}{=}\PY{n}{SparseContinuityOperator}\PY{p}{(}\PY{l+m+mi}{0}\PY{o}{*}\PY{n}{difegrid}\PY{p}{,}\PY{n}{dx}\PY{p}{,}\PY{n}{dt}\PY{p}{,}\PY{n}{efield}\PY{o}{*}\PY{n}{mobegrid}\PY{p}{)}
             \PY{n}{difOperatorIon1}\PY{o}{=}\PY{n}{SparseContinuityOperator}\PY{p}{(}\PY{l+m+mi}{0}\PY{o}{*}\PY{n}{difigrid}\PY{p}{,}\PY{n}{dx}\PY{p}{,}\PY{n}{dt}\PY{p}{,}\PY{n}{efield}\PY{o}{*}\PY{n}{mobigrid}\PY{p}{)}
             \PY{n}{ndensity}\PY{p}{[}\PY{l+m+mi}{0}\PY{p}{,}\PY{l+m+mi}{1}\PY{p}{:}\PY{o}{\PYZhy{}}\PY{l+m+mi}{1}\PY{p}{]}\PY{o}{=}\PY{n}{la}\PY{o}{.}\PY{n}{spsolve}\PY{p}{(}\PY{n}{difOperatorElectron1}\PY{p}{,}\PY{n}{ndentemp}\PY{p}{[}\PY{l+m+mi}{0}\PY{p}{]}\PY{p}{)}\PY{p}{[}\PY{l+m+mi}{1}\PY{p}{:}\PY{o}{\PYZhy{}}\PY{l+m+mi}{1}\PY{p}{]}\PY{o}{.}\PY{n}{copy}\PY{p}{(}\PY{p}{)}
             \PY{n}{ndensity}\PY{p}{[}\PY{l+m+mi}{1}\PY{p}{,}\PY{l+m+mi}{1}\PY{p}{:}\PY{o}{\PYZhy{}}\PY{l+m+mi}{1}\PY{p}{]}\PY{o}{=}\PY{n}{la}\PY{o}{.}\PY{n}{spsolve}\PY{p}{(}\PY{n}{difOperatorIon1}\PY{p}{,}\PY{n}{ndentemp}\PY{p}{[}\PY{l+m+mi}{1}\PY{p}{]}\PY{p}{)}\PY{p}{[}\PY{l+m+mi}{1}\PY{p}{:}\PY{o}{\PYZhy{}}\PY{l+m+mi}{1}\PY{p}{]}\PY{o}{.}\PY{n}{copy}\PY{p}{(}\PY{p}{)}
             \PY{n}{ndensity}\PY{p}{[}\PY{n}{ndensity}\PY{o}{\PYZlt{}}\PY{l+m+mi}{1000}\PY{p}{]}\PY{o}{=}\PY{l+m+mf}{1000.}
         
         
             \PY{c+c1}{\PYZsh{}Source and sink}
             \PY{c+c1}{\PYZsh{}=============================================================================}
             \PY{c+c1}{\PYZsh{}source}
             \PY{n}{etemperature}\PY{o}{=}\PY{n+nb}{abs}\PY{p}{(}\PY{p}{(}\PY{n}{ee}\PY{o}{/}\PY{n}{Kboltz}\PY{p}{)}\PY{o}{*}\PY{n}{difegrid}\PY{o}{/}\PY{n}{mobegrid}\PY{p}{)}
             \PY{n}{reverserate}\PY{o}{=}\PY{l+m+mf}{8.1}\PY{o}{*}\PY{l+m+mi}{10}\PY{o}{*}\PY{o}{*}\PY{p}{(}\PY{o}{\PYZhy{}}\PY{l+m+mi}{13}\PY{p}{)}\PY{o}{*}\PY{p}{(}\PY{n}{etemperature}\PY{o}{/}\PY{l+m+mi}{300}\PY{p}{)}\PY{o}{*}\PY{o}{*}\PY{p}{(}\PY{o}{\PYZhy{}}\PY{l+m+mf}{0.64}\PY{p}{)}
             \PY{n}{decrementt}\PY{o}{=}\PY{n}{reverserate}\PY{p}{[}\PY{l+m+mi}{1}\PY{p}{:}\PY{o}{\PYZhy{}}\PY{l+m+mi}{1}\PY{p}{]}\PY{o}{*}\PY{n}{ndensity}\PY{p}{[}\PY{l+m+mi}{0}\PY{p}{,}\PY{l+m+mi}{1}\PY{p}{:}\PY{o}{\PYZhy{}}\PY{l+m+mi}{1}\PY{p}{]}\PY{o}{*}\PY{n}{ndensity}\PY{p}{[}\PY{l+m+mi}{1}\PY{p}{,}\PY{l+m+mi}{1}\PY{p}{:}\PY{o}{\PYZhy{}}\PY{l+m+mi}{1}\PY{p}{]}\PY{o}{*}\PY{n}{dt}
             \PY{n}{ndensity}\PY{p}{[}\PY{l+m+mi}{0}\PY{p}{,}\PY{l+m+mi}{1}\PY{p}{:}\PY{o}{\PYZhy{}}\PY{l+m+mi}{1}\PY{p}{]}\PY{o}{=}\PY{n}{ndensity}\PY{p}{[}\PY{l+m+mi}{0}\PY{p}{,}\PY{l+m+mi}{1}\PY{p}{:}\PY{o}{\PYZhy{}}\PY{l+m+mi}{1}\PY{p}{]}\PY{o}{\PYZhy{}}\PY{n}{decrementt}
             \PY{n}{ndensity}\PY{p}{[}\PY{l+m+mi}{1}\PY{p}{,}\PY{l+m+mi}{1}\PY{p}{:}\PY{o}{\PYZhy{}}\PY{l+m+mi}{1}\PY{p}{]}\PY{o}{=}\PY{n}{ndensity}\PY{p}{[}\PY{l+m+mi}{1}\PY{p}{,}\PY{l+m+mi}{1}\PY{p}{:}\PY{o}{\PYZhy{}}\PY{l+m+mi}{1}\PY{p}{]}\PY{o}{\PYZhy{}}\PY{n}{decrementt}
             
             \PY{n}{ndensity}\PY{p}{[}\PY{l+m+mi}{0}\PY{p}{,}\PY{l+m+mi}{1}\PY{p}{:}\PY{o}{\PYZhy{}}\PY{l+m+mi}{1}\PY{p}{]}\PY{o}{=}\PY{n}{ndensity}\PY{p}{[}\PY{l+m+mi}{0}\PY{p}{,}\PY{l+m+mi}{1}\PY{p}{:}\PY{o}{\PYZhy{}}\PY{l+m+mi}{1}\PY{p}{]}\PY{o}{+}\PY{l+m+mf}{1.0}\PY{o}{*}\PY{p}{(}  \PY{n}{sourceegrid}\PY{p}{[}\PY{l+m+mi}{1}\PY{p}{:}\PY{o}{\PYZhy{}}\PY{l+m+mi}{1}\PY{p}{]}\PY{o}{*}\PY{n+nb}{abs}\PY{p}{(}\PY{n}{ndensity}\PY{p}{[}\PY{l+m+mi}{0}\PY{p}{,}\PY{l+m+mi}{1}\PY{p}{:}\PY{o}{\PYZhy{}}\PY{l+m+mi}{1}\PY{p}{]}\PY{p}{)}\PY{o}{*}\PY{n+nb}{abs}\PY{p}{(}\PY{p}{(}\PY{n}{gasdens}\PY{o}{\PYZhy{}}\PY{n}{ndensity}\PY{p}{[}\PY{l+m+mi}{1}\PY{p}{,}\PY{l+m+mi}{1}\PY{p}{:}\PY{o}{\PYZhy{}}\PY{l+m+mi}{1}\PY{p}{]}\PY{p}{)}\PY{o}{/}\PY{n}{gasdens}\PY{p}{)}\PY{p}{)} \PY{o}{*} \PY{n}{dt} 
             \PY{n}{ndensity}\PY{p}{[}\PY{l+m+mi}{1}\PY{p}{,}\PY{l+m+mi}{1}\PY{p}{:}\PY{o}{\PYZhy{}}\PY{l+m+mi}{1}\PY{p}{]}\PY{o}{=}\PY{n}{ndensity}\PY{p}{[}\PY{l+m+mi}{1}\PY{p}{,}\PY{l+m+mi}{1}\PY{p}{:}\PY{o}{\PYZhy{}}\PY{l+m+mi}{1}\PY{p}{]}\PY{o}{+}\PY{l+m+mf}{1.0}\PY{o}{*}\PY{p}{(}  \PY{n}{sourceegrid}\PY{p}{[}\PY{l+m+mi}{1}\PY{p}{:}\PY{o}{\PYZhy{}}\PY{l+m+mi}{1}\PY{p}{]}\PY{o}{*}\PY{n+nb}{abs}\PY{p}{(}\PY{n}{ndensity}\PY{p}{[}\PY{l+m+mi}{0}\PY{p}{,}\PY{l+m+mi}{1}\PY{p}{:}\PY{o}{\PYZhy{}}\PY{l+m+mi}{1}\PY{p}{]}\PY{p}{)}\PY{o}{*}\PY{n+nb}{abs}\PY{p}{(}\PY{p}{(}\PY{n}{gasdens}\PY{o}{\PYZhy{}}\PY{n}{ndensity}\PY{p}{[}\PY{l+m+mi}{1}\PY{p}{,}\PY{l+m+mi}{1}\PY{p}{:}\PY{o}{\PYZhy{}}\PY{l+m+mi}{1}\PY{p}{]}\PY{p}{)}\PY{o}{/}\PY{n}{gasdens}\PY{p}{)}\PY{p}{)} \PY{o}{*} \PY{n}{dt}
             
             \PY{c+c1}{\PYZsh{}charge accumulation at surface of dielectric}
             \PY{c+c1}{\PYZsh{}=========================================================================================}
             \PY{n}{efluxleft}\PY{o}{=}\PY{o}{\PYZhy{}}\PY{l+m+mf}{0.5}\PY{o}{*}\PY{p}{(}\PY{n}{ndensity}\PY{p}{[}\PY{l+m+mi}{0}\PY{p}{,}\PY{l+m+mi}{1}\PY{p}{]}\PY{o}{+}\PY{n}{ndensity}\PY{p}{[}\PY{l+m+mi}{0}\PY{p}{,}\PY{l+m+mi}{2}\PY{p}{]}\PY{p}{)}\PY{o}{*}\PY{p}{(}\PY{n}{mobegrid}\PY{p}{[}\PY{l+m+mi}{1}\PY{p}{]}\PY{p}{)}\PY{o}{*}\PY{n}{efield}\PY{p}{[}\PY{l+m+mi}{1}\PY{p}{]}
             \PY{k}{if} \PY{n}{efluxleft}\PY{o}{\PYZlt{}}\PY{l+m+mi}{0}\PY{p}{:}
                 \PY{n}{efluxleft}\PY{o}{=}\PY{l+m+mi}{0}
             \PY{n}{sig\PYZus{}e\PYZus{}left}\PY{o}{=} \PY{n}{sig\PYZus{}e\PYZus{}left}\PY{o}{+}\PY{n}{dt}\PY{o}{*}\PY{p}{(}\PY{n}{efluxleft}\PY{o}{\PYZhy{}}\PY{l+m+mi}{10}\PY{o}{*}\PY{n}{sig\PYZus{}e\PYZus{}left}\PY{o}{\PYZhy{}}\PY{l+m+mi}{10}\PY{o}{*}\PY{o}{*}\PY{p}{(}\PY{o}{\PYZhy{}}\PY{l+m+mi}{10}\PY{p}{)}\PY{o}{*}\PY{n}{sig\PYZus{}e\PYZus{}left}\PY{o}{*}\PY{n}{sig\PYZus{}e\PYZus{}right}\PY{p}{)}
             
             \PY{n}{efluxright}\PY{o}{=}\PY{l+m+mf}{0.5}\PY{o}{*}\PY{p}{(}\PY{n}{ndensity}\PY{p}{[}\PY{l+m+mi}{0}\PY{p}{,}\PY{o}{\PYZhy{}}\PY{l+m+mi}{2}\PY{p}{]}\PY{o}{+}\PY{n}{ndensity}\PY{p}{[}\PY{l+m+mi}{0}\PY{p}{,}\PY{o}{\PYZhy{}}\PY{l+m+mi}{3}\PY{p}{]}\PY{p}{)}\PY{o}{*}\PY{p}{(}\PY{n}{mobegrid}\PY{p}{[}\PY{o}{\PYZhy{}}\PY{l+m+mi}{2}\PY{p}{]}\PY{p}{)}\PY{o}{*}\PY{n}{efield}\PY{p}{[}\PY{o}{\PYZhy{}}\PY{l+m+mi}{2}\PY{p}{]}
             \PY{k}{if} \PY{n}{efluxright}\PY{o}{\PYZlt{}}\PY{l+m+mi}{0}\PY{p}{:}
                 \PY{n}{efluxright}\PY{o}{=}\PY{l+m+mi}{0}
             \PY{n}{sig\PYZus{}e\PYZus{}right}\PY{o}{=}\PY{n}{sig\PYZus{}e\PYZus{}right}\PY{o}{+}\PY{n}{dt}\PY{o}{*}\PY{p}{(}\PY{n}{efluxright}\PY{o}{\PYZhy{}}\PY{l+m+mi}{10}\PY{o}{*}\PY{n}{sig\PYZus{}e\PYZus{}right}\PY{o}{\PYZhy{}}\PY{l+m+mi}{10}\PY{o}{*}\PY{o}{*}\PY{p}{(}\PY{o}{\PYZhy{}}\PY{l+m+mi}{10}\PY{p}{)}\PY{o}{*}\PY{n}{sig\PYZus{}e\PYZus{}left}\PY{o}{*}\PY{n}{sig\PYZus{}e\PYZus{}right}\PY{p}{)}
             
             \PY{n}{ifluxleft}\PY{o}{=}\PY{o}{\PYZhy{}}\PY{p}{(}\PY{l+m+mi}{1}\PY{o}{+}\PY{l+m+mf}{0.01}\PY{p}{)}\PY{o}{*}\PY{l+m+mf}{0.5}\PY{o}{*}\PY{p}{(}\PY{n}{ndensity}\PY{p}{[}\PY{l+m+mi}{1}\PY{p}{,}\PY{l+m+mi}{1}\PY{p}{]}\PY{o}{+}\PY{n}{ndensity}\PY{p}{[}\PY{l+m+mi}{1}\PY{p}{,}\PY{l+m+mi}{2}\PY{p}{]}\PY{p}{)}\PY{o}{*}\PY{p}{(}\PY{n}{mobigrid}\PY{p}{[}\PY{l+m+mi}{1}\PY{p}{]}\PY{p}{)}\PY{o}{*}\PY{n}{efield}\PY{p}{[}\PY{l+m+mi}{1}\PY{p}{]}
             \PY{k}{if} \PY{n}{ifluxleft}\PY{o}{\PYZlt{}}\PY{l+m+mi}{0}\PY{p}{:}
                 \PY{n}{ifluxleft}\PY{o}{=}\PY{l+m+mi}{0}
             \PY{n}{sig\PYZus{}i\PYZus{}left}\PY{o}{=}\PY{l+m+mi}{1}\PY{o}{*}\PY{p}{(}\PY{n}{sig\PYZus{}i\PYZus{}left}\PY{o}{+}\PY{n}{dt}\PY{o}{*}\PY{p}{(}\PY{n}{ifluxleft}\PY{o}{\PYZhy{}}\PY{l+m+mi}{10}\PY{o}{*}\PY{o}{*}\PY{p}{(}\PY{o}{\PYZhy{}}\PY{l+m+mi}{10}\PY{p}{)}\PY{o}{*}\PY{n}{sig\PYZus{}i\PYZus{}left}\PY{o}{*}\PY{n}{sig\PYZus{}i\PYZus{}right}\PY{p}{)}\PY{p}{)}
             \PY{n}{ifluxright}\PY{o}{=}\PY{p}{(}\PY{l+m+mi}{1}\PY{o}{+}\PY{l+m+mf}{0.01}\PY{p}{)}\PY{o}{*}\PY{l+m+mf}{0.5}\PY{o}{*}\PY{p}{(}\PY{n}{ndensity}\PY{p}{[}\PY{l+m+mi}{1}\PY{p}{,}\PY{o}{\PYZhy{}}\PY{l+m+mi}{2}\PY{p}{]}\PY{o}{+}\PY{n}{ndensity}\PY{p}{[}\PY{l+m+mi}{1}\PY{p}{,}\PY{o}{\PYZhy{}}\PY{l+m+mi}{3}\PY{p}{]}\PY{p}{)}\PY{o}{*}\PY{p}{(}\PY{n}{mobigrid}\PY{p}{[}\PY{o}{\PYZhy{}}\PY{l+m+mi}{2}\PY{p}{]}\PY{p}{)}\PY{o}{*}\PY{n}{efield}\PY{p}{[}\PY{o}{\PYZhy{}}\PY{l+m+mi}{2}\PY{p}{]}
             \PY{k}{if} \PY{n}{ifluxright}\PY{o}{\PYZlt{}}\PY{l+m+mi}{0}\PY{p}{:}
                 \PY{n}{ifluxright}\PY{o}{=}\PY{l+m+mi}{0}
             \PY{n}{sig\PYZus{}i\PYZus{}right}\PY{o}{=}\PY{l+m+mi}{1}\PY{o}{*}\PY{p}{(}\PY{n}{sig\PYZus{}i\PYZus{}right}\PY{o}{+}\PY{n}{dt}\PY{o}{*}\PY{p}{(}\PY{n}{ifluxright}\PY{o}{\PYZhy{}}\PY{l+m+mi}{10}\PY{o}{*}\PY{o}{*}\PY{p}{(}\PY{o}{\PYZhy{}}\PY{l+m+mi}{10}\PY{p}{)}\PY{o}{*}\PY{n}{sig\PYZus{}i\PYZus{}left}\PY{o}{*}\PY{n}{sig\PYZus{}i\PYZus{}right}\PY{p}{)}\PY{p}{)}
             \PY{c+c1}{\PYZsh{}print (sig\PYZus{}e\PYZus{}left,sig\PYZus{}e\PYZus{}right,sig\PYZus{}i\PYZus{}left,sig\PYZus{}i\PYZus{}right)}
             
             \PY{n}{ndensity}\PY{p}{[}\PY{l+m+mi}{0}\PY{p}{,}\PY{l+m+mi}{0}\PY{p}{]}\PY{o}{=}\PY{l+m+mf}{0.5}\PY{o}{*}\PY{p}{(}\PY{n}{ndensity}\PY{p}{[}\PY{l+m+mi}{0}\PY{p}{,}\PY{l+m+mi}{1}\PY{p}{]}\PY{o}{+}\PY{n}{ndensity}\PY{p}{[}\PY{l+m+mi}{0}\PY{p}{,}\PY{l+m+mi}{2}\PY{p}{]}\PY{p}{)}\PY{o}{+}\PY{l+m+mf}{1.}\PY{o}{*}\PY{p}{(}\PY{n}{sig\PYZus{}e\PYZus{}left}\PY{p}{)}\PY{o}{/}\PY{n}{dx}
             \PY{n}{ndensity}\PY{p}{[}\PY{l+m+mi}{0}\PY{p}{,}\PY{o}{\PYZhy{}}\PY{l+m+mi}{1}\PY{p}{]}\PY{o}{=}\PY{l+m+mf}{0.5}\PY{o}{*}\PY{p}{(}\PY{n}{ndensity}\PY{p}{[}\PY{l+m+mi}{0}\PY{p}{,}\PY{o}{\PYZhy{}}\PY{l+m+mi}{2}\PY{p}{]}\PY{o}{+}\PY{n}{ndensity}\PY{p}{[}\PY{l+m+mi}{0}\PY{p}{,}\PY{o}{\PYZhy{}}\PY{l+m+mi}{3}\PY{p}{]}\PY{p}{)}\PY{o}{+}\PY{l+m+mf}{1.}\PY{o}{*}\PY{p}{(}\PY{n}{sig\PYZus{}e\PYZus{}right}\PY{p}{)}\PY{o}{/}\PY{n}{dx}    
             \PY{n}{ndensity}\PY{p}{[}\PY{l+m+mi}{1}\PY{p}{,}\PY{l+m+mi}{0}\PY{p}{]}\PY{o}{=}\PY{l+m+mf}{0.5}\PY{o}{*}\PY{p}{(}\PY{n}{ndensity}\PY{p}{[}\PY{l+m+mi}{1}\PY{p}{,}\PY{l+m+mi}{1}\PY{p}{]}\PY{o}{+}\PY{n}{ndensity}\PY{p}{[}\PY{l+m+mi}{1}\PY{p}{,}\PY{l+m+mi}{2}\PY{p}{]}\PY{p}{)}\PY{o}{+}\PY{l+m+mf}{1.}\PY{o}{*}\PY{p}{(}\PY{n}{sig\PYZus{}i\PYZus{}left}\PY{p}{)}\PY{o}{/}\PY{n}{dx}
             \PY{n}{ndensity}\PY{p}{[}\PY{l+m+mi}{1}\PY{p}{,}\PY{o}{\PYZhy{}}\PY{l+m+mi}{1}\PY{p}{]}\PY{o}{=}\PY{l+m+mf}{0.5}\PY{o}{*}\PY{p}{(}\PY{n}{ndensity}\PY{p}{[}\PY{l+m+mi}{1}\PY{p}{,}\PY{o}{\PYZhy{}}\PY{l+m+mi}{2}\PY{p}{]}\PY{o}{+}\PY{n}{ndensity}\PY{p}{[}\PY{l+m+mi}{1}\PY{p}{,}\PY{o}{\PYZhy{}}\PY{l+m+mi}{3}\PY{p}{]}\PY{p}{)}\PY{o}{+}\PY{l+m+mf}{1.}\PY{o}{*}\PY{p}{(}\PY{n}{sig\PYZus{}i\PYZus{}right}\PY{p}{)}\PY{o}{/}\PY{n}{dx}    
             
             \PY{c+c1}{\PYZsh{}adaptive time stepping==================================================================}
             \PY{n}{mobsta}\PY{o}{=}\PY{l+m+mf}{0.5}\PY{o}{*}\PY{p}{(}\PY{n}{mobegrid}\PY{p}{[}\PY{l+m+mi}{1}\PY{p}{:}\PY{p}{]}\PY{o}{+}\PY{n}{mobegrid}\PY{p}{[}\PY{p}{:}\PY{o}{\PYZhy{}}\PY{l+m+mi}{1}\PY{p}{]}\PY{p}{)}
             \PY{n}{difsta}\PY{o}{=}\PY{l+m+mf}{0.5}\PY{o}{*}\PY{p}{(}\PY{n}{difegrid}\PY{p}{[}\PY{l+m+mi}{1}\PY{p}{:}\PY{p}{]}\PY{o}{+}\PY{n}{difegrid}\PY{p}{[}\PY{p}{:}\PY{o}{\PYZhy{}}\PY{l+m+mi}{1}\PY{p}{]}\PY{p}{)}
             \PY{n}{eefsta}\PY{o}{=}\PY{l+m+mf}{0.5}\PY{o}{*}\PY{p}{(}\PY{n}{efield}\PY{p}{[}\PY{l+m+mi}{1}\PY{p}{:}\PY{p}{]}\PY{o}{+}\PY{n}{efield}\PY{p}{[}\PY{p}{:}\PY{o}{\PYZhy{}}\PY{l+m+mi}{1}\PY{p}{]}\PY{p}{)}\PY{o}{/}\PY{n}{townsendunit}
             \PY{n}{gstability}\PY{o}{=}\PY{n+nb}{max}\PY{p}{(}\PY{n+nb}{abs}\PY{p}{(}\PY{n}{mobsta}\PY{o}{*}\PY{p}{(}\PY{n}{efield}\PY{p}{[}\PY{l+m+mi}{1}\PY{p}{:}\PY{p}{]}\PY{o}{\PYZhy{}}\PY{n}{efield}\PY{p}{[}\PY{p}{:}\PY{o}{\PYZhy{}}\PY{l+m+mi}{1}\PY{p}{]}\PY{p}{)}\PY{o}{/}\PY{p}{(}\PY{n}{dx}\PY{o}{*}\PY{n}{townsendunit}\PY{p}{)}\PY{o}{+}\PY{n}{eefsta}\PY{o}{*}\PY{p}{(}\PY{n}{mobegrid}\PY{p}{[}\PY{l+m+mi}{1}\PY{p}{:}\PY{p}{]}\PY{o}{\PYZhy{}}\PY{n}{mobegrid}\PY{p}{[}\PY{p}{:}\PY{o}{\PYZhy{}}\PY{l+m+mi}{1}\PY{p}{]}\PY{p}{)}\PY{o}{/}\PY{n}{dx}\PY{o}{+}\PY{n}{mobsta}\PY{o}{*}\PY{n}{eefsta}\PY{o}{/}\PY{p}{(}\PY{l+m+mi}{2}\PY{o}{*}\PY{n}{dx}\PY{p}{)}      \PY{o}{+}\PY{l+m+mi}{4}\PY{o}{*}\PY{n}{difsta}\PY{o}{/}\PY{p}{(}\PY{n}{dx}\PY{o}{*}\PY{n}{dx}\PY{p}{)} \PY{o}{+}\PY{p}{(}\PY{n}{difegrid}\PY{p}{[}\PY{l+m+mi}{1}\PY{p}{:}\PY{p}{]}\PY{o}{\PYZhy{}}\PY{n}{difegrid}\PY{p}{[}\PY{p}{:}\PY{o}{\PYZhy{}}\PY{l+m+mi}{1}\PY{p}{]}\PY{p}{)}\PY{o}{/}\PY{p}{(}\PY{n}{dx}\PY{o}{*}\PY{n}{dx}\PY{p}{)}\PY{p}{)}\PY{p}{)}
             \PY{n}{dt}\PY{o}{=}\PY{l+m+mf}{10e\PYZhy{}9}
             
             \PY{c+c1}{\PYZsh{}printing and saving the data============================================================}
             \PY{c+c1}{\PYZsh{}f = open(\PYZsq{}finaldata/new.txt\PYZsq{}, \PYZsq{}ab\PYZsq{})}
             \PY{k}{if} \PY{p}{(}\PY{n}{tymeStep} \PY{o}{\PYZpc{}} \PY{l+m+mi}{1000} \PY{o}{==} \PY{l+m+mi}{0} \PY{p}{)}\PY{p}{:} \PY{c+c1}{\PYZsh{}and leftPot\PYZgt{}1080 ):}
                 \PY{c+c1}{\PYZsh{}print (efluxleft,ifluxleft,efluxright,ifluxright)}
                 \PY{n+nb}{print}\PY{p}{(}\PY{n+nb}{max}\PY{p}{(}\PY{n}{etemperature}\PY{p}{)}\PY{p}{)}
                 \PY{c+c1}{\PYZsh{}print()}
                 \PY{c+c1}{\PYZsh{}np.savetxt(f, ndensity[0,:])}
                 \PY{c+c1}{\PYZsh{}np.savetxt(f, ndensity[1,:])}
                 \PY{c+c1}{\PYZsh{}np.savetxt(f, netcharge)}
                 \PY{c+c1}{\PYZsh{}np.savetxt(f, efield)}
                 \PY{c+c1}{\PYZsh{}np.savetxt(f, potentl);}
                 \PY{c+c1}{\PYZsh{}if (tymeStep \PYZpc{} 1000 == 0 ): \PYZsh{}and leftPot\PYZgt{}1080 ):}
                 \PY{c+c1}{\PYZsh{}f.close()}
                 \PY{n}{fourPlots}\PY{p}{(}\PY{p}{(}\PY{n}{ndensity}\PY{p}{[}\PY{l+m+mi}{0}\PY{p}{,}\PY{l+m+mi}{5}\PY{p}{:}\PY{o}{\PYZhy{}}\PY{l+m+mi}{5}\PY{p}{]}\PY{p}{,}\PY{n}{ndensity}\PY{p}{[}\PY{l+m+mi}{1}\PY{p}{,}\PY{l+m+mi}{5}\PY{p}{:}\PY{o}{\PYZhy{}}\PY{l+m+mi}{5}\PY{p}{]}\PY{p}{)}\PY{p}{,}\PY{l+s+s1}{\PYZsq{}}\PY{l+s+s1}{ndensity}\PY{l+s+s1}{\PYZsq{}}\PY{p}{,}\PY{n}{netcharge}\PY{p}{,}\PY{l+s+s1}{\PYZsq{}}\PY{l+s+s1}{netcharge}\PY{l+s+s1}{\PYZsq{}}\PY{p}{,}\PY{n}{potentl}\PY{p}{,}\PY{l+s+s1}{\PYZsq{}}\PY{l+s+s1}{electric potential}\PY{l+s+s1}{\PYZsq{}}\PY{p}{,}\PY{n}{efield}\PY{p}{,}\PY{l+s+s1}{\PYZsq{}}\PY{l+s+s1}{\PYZsq{}}\PY{p}{)}
         \PY{c+c1}{\PYZsh{}        print(ndensity[0,:5],ndensity[1,:5])}
         \PY{c+c1}{\PYZsh{}        print(ndensity[0,\PYZhy{}5:],ndensity[1,\PYZhy{}5:])}
                 \PY{c+c1}{\PYZsh{}print(\PYZsq{}eleft\PYZsq{}, efluxleft, \PYZsq{}eritht\PYZsq{}, efluxright, \PYZsq{}ileft\PYZsq{}, ifluxleft, \PYZsq{}iright\PYZsq{}, ifluxright)}
                 \PY{c+c1}{\PYZsh{}print(\PYZsq{}eleft\PYZsq{}, efield[2], \PYZsq{}eritht\PYZsq{}, efield[\PYZhy{}2], \PYZsq{}ileft\PYZsq{}, efield[2], \PYZsq{}iright\PYZsq{}, efield[\PYZhy{}2])}
                 \PY{c+c1}{\PYZsh{}print(ndensity[0,100],ndensity[1,100])}
                 \PY{n+nb}{print}\PY{p}{(}\PY{n}{dt}\PY{p}{)}
\end{Verbatim}


    \begin{Verbatim}[commandchars=\\\{\}]
1933.4037137092616

    \end{Verbatim}

    \begin{center}
    \adjustimage{max size={0.9\linewidth}{0.9\paperheight}}{output_19_1.png}
    \end{center}
    { \hspace*{\fill} \\}
    
    \begin{Verbatim}[commandchars=\\\{\}]
1e-08
3332.6414717751886

    \end{Verbatim}

    \begin{center}
    \adjustimage{max size={0.9\linewidth}{0.9\paperheight}}{output_19_3.png}
    \end{center}
    { \hspace*{\fill} \\}
    
    \begin{Verbatim}[commandchars=\\\{\}]
1e-08

    \end{Verbatim}

    \subsection{Results and discussions}\label{results-and-discussions}

\paragraph{Temporal variation of electron density inside the
reactor}\label{temporal-variation-of-electron-density-inside-the-reactor}

The solution to the above continuity gives the value of the electron and
ion densities inside the plasma reactor. The charged particles are
either advected or diffused inside the reactor. There is not dominant
effect of recombination in low temperature plasma device. The loss of
charged particles occurs only on the boundaries. It can be seen from the
equation of charge accumulation on the dielectric surface that there is
a term that accounts for the recombination of ions and electrons on the
surface.

The formation of charged particles corresponds to the chemical reaction
of the gases. The beam of electron that is accelerated towards the
electric field collides with the netural molecules with sufficient
energy to cause the ionization. The ionized electron ion pairs follow
the path towards the electrodes. While the electron moves towards the
electrodes, it collides with more neutral molecules giving rise to more
electron/ion pairs.

It can be seen from the figure that the value of electron/ion density is
almost zero in the beginning. It incerases gradually and finally reaches
to the maximum value where it becomes stable.

\paragraph{Optical Characterization}\label{optical-characterization}

The optical characterization of the argon dielectric barrier discharge
system was done with the help of Optical Emission Spectrometer (OES)
that involves the detection of characterstic emission spectral lines
generated by excited atoms and ions inside the plasma. The photon
generated by the plasma is said to be the combination of all those
spectral lines. Individual species can have a number of lines. The OES
is a simple procedure and produces no perterbutation in plasma. It is
used to detect the types of gaseous species present in the discharge.
The OES data is also used to determine the electron density of the
plasma by using the intensity ratio method {[}..{]} that is based on the
relative intensities of the atomic lines.

Since the cross section of the plasma reactor consisted of the
polycarbonate substance which absorbs visible light of certain
wavelength a hole was drilled on the reactor so that the fiber optical
cable of the emission spectrometer could be directly inserted inside the
reactor to take more accurate emission data. The emission spectrometer
simply gives the intensity vs wavelength data.

The above spectra suggest the presence of Ar-I lines inside the plasma
reactor. Since the intensity ration method is based on the relative
intensity, the above OES data is only useful to confirm the presence of
argon-I but not useful to calculate the electron density by using the
intensity ratio method.

    \subsubsection{Electrical
Characterization}\label{electrical-characterization}

The electrodes of the reactor are connected to a high voltage power
source and a HV prove is connected in parallel to the electrodes. High
voltage probe is connected to Cathode ray oscilloscope that gives the
value of voltage accross the reactor. Whereas the shunt resistor is
connected in series to the reactor. The value of voltage accross the
shunt resistor is also obtained using the CRO. Reactor current is
proportional to the voltage accross the shunt resistor as the resistance
of the resistor is constant.

The above graph shows the temporal variation of current in the plasma
reactor for four complete cycles of AC voltage. Very small amount of
current is flowing accross the reactor. The sudden rise in the value of
current as it reaches the maxima is due to the micro-discharge
phenemenon. The plasma is formed only at that instant of time.

\subparagraph{Variation of electric field inside the
reactor}\label{variation-of-electric-field-inside-the-reactor}

The major advantage of using dielectric barrier discharge instead of
traditional plasma reactor is that there is the accumulation of charged
particles on the surface of the dielectric which in return works against
the development of field inside the reactor. It can be confirmed from
the above graph that there is not significant increase in the value of
electric field inside the reactor which contributes in the nature of
discharge by preventing the formation of arc. The discharge becomes more
glow type than the arc type. Electric field is measured in plasma as
Townsend units.

    \paragraph{Variation of electric
potential}\label{variation-of-electric-potential}

The right electrode is grounded (0volts) while the left electrode is
connected to a high frequency AC source \((V_o Sin(\omega t))\).
Following image shows the variation of electric potential inside the
reactor. Like the electric field there is no sharp variation in the
nature of the electric potential inside the reactor. The profile


    % Add a bibliography block to the postdoc
    
    
    
    \end{document}
